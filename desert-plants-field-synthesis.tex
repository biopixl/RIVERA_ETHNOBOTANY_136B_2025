\documentclass[10pt,letterpaper,twocolumn]{article}
\usepackage[utf8]{inputenc}
\usepackage[T1]{fontenc}
\usepackage[margin=0.6in]{geometry}
\usepackage{enumitem}
\usepackage{booktabs}
\usepackage{xcolor}
\usepackage{titlesec}
\usepackage{setspace}
\usepackage{graphicx}
\usepackage{wrapfig}
\usepackage{caption}

% Graphics path
\graphicspath{{figures/extracted/}}

% Spacing
\setlength{\parskip}{4pt}
\setlength{\parindent}{0pt}
\titlespacing*{\section}{0pt}{10pt}{4pt}
\titlespacing*{\subsection}{0pt}{7pt}{3pt}

% Section styling
\titleformat{\section}{\normalfont\large\bfseries\scshape}{\thesection}{0.5em}{}
\titleformat{\subsection}{\normalfont\bfseries}{\thesubsection}{0.5em}{}

\begin{document}

\begin{center}
{\Large\scshape Desert Plant Healing, Rituals, and Material Culture}\\[4pt]
{\small Indigenous Ethnobotany of the Saline Valley Region}\\[2pt]
{\footnotesize Ge136 Field Resource}
\end{center}

\vspace{-4pt}
\noindent\rule{\columnwidth}{0.4pt}
\vspace{4pt}

The Indigenous peoples of the Saline Valley region---Kawaiisu, Tubatulabal, Western Shoshone, Eastern Mono, and Northern Paiute---developed ethnobotanical knowledge systems whose sophistication is increasingly validated by contemporary pharmacological research. Their traditions reveal integrated worldviews in which healing, spiritual practice, and material production formed a continuous fabric, with empirical observations accumulated across millennia now finding mechanistic explanation through modern analytical chemistry. Daniel Moerman's \textit{Native American Ethnobotany} identifies big sagebrush (\textit{Artemisia tridentata}) among the ten plants with the greatest number of documented uses across North American tribes---a distinction comprehensible upon entering the Great Basin, where sagebrush dominates from valley floors to mid-elevation slopes and where its volatile aromatics saturate the air after rain.

%==============================================================
\section{Healing Traditions}
%==============================================================

Traditional Shoshone epistemology held that food and medicine were inseparable categories. This perspective finds support in contemporary nutritional science: in environments where over thirty edible species required intimate knowledge of habitat, seasonality, and preparation, the distinction between nutritional and therapeutic properties dissolves under empirical scrutiny.

\subsection{Respiratory and Febrifuge Applications}

\begin{wrapfigure}{r}{0.45\columnwidth}
\centering
\includegraphics[width=0.42\columnwidth]{sagebrush-01}
\caption*{\scriptsize\textit{Artemisia tridentata}. USDA.}
\end{wrapfigure}

Sagebrush served as the primary respiratory therapeutic across the region. Among the Goshute of Utah, leaf infusions treated fevers, colds, and coughs with sufficient efficacy that the plant functioned as a general remedy for ``most difficulties or disease.'' The Navajo employed sagebrush vapor inhalation for headaches, while Zuni practitioners administered leaf infusions both internally for colds and externally as analgesic washes.

Gas chromatography-mass spectrometry analysis reveals the phytochemical basis for these applications: sagebrush essential oil contains approximately 40\% \textit{l}-camphor, with significant contributions from $\alpha$-pinene, 1,8-cineole, and thujone (Adams et al., 2012). Camphor functions as an agonist for TRPV3 heat-sensitive receptors and an antagonist for TRPA1 cold-sensitive receptors, explaining the sensation of warmth and relief reported in traditional use. Rahman et al. (2017) demonstrated that these monoterpenoids exhibit anti-inflammatory activity through inhibition of nitric oxide (NO) and prostaglandin E\textsubscript{2} (PGE\textsubscript{2}) production via suppression of iNOS and NF-$\kappa$B expression pathways.

\textit{Field preparation}: Collect terminal leaf clusters from mature plants, avoiding damaged or insect-infested material. Traditional infusion requires steeping 2--3 tablespoons of fresh or dried leaves in near-boiling water for 10--15 minutes. The characteristic camphoraceous aroma intensifies during steeping, indicating volatile release. For vapor inhalation, crush fresh leaves and hold beneath a cloth draped over the head with a bowl of hot water---a method that maximizes monoterpene delivery to respiratory mucosa.

\subsection{Orthopedic and Analgesic Practice}

Creosote bush (\textit{Larrea tridentata}) dominated orthopedic medicine throughout its range. The Diegue\~{n}o prepared leaf decoctions as baths for rheumatism and arthritis, a practice paralleled by Kawaiisu use of leaf washes for general musculoskeletal pain. The Papago developed a distinct technique: holding afflicted feet above smoldering creosote branches, combining localized heat with volatile compound delivery.

These applications exploit nordihydroguaiaretic acid (NDGA), a lignan comprising approximately 10\% of leaf dry weight and 80\% of total phenolic content in the resin. Rahman et al. (2011) demonstrated that NDGA pretreatment mitigates cutaneous lipid peroxidation and inhibits hydrogen peroxide production in inflammatory models. The mechanism involves potent antioxidant activity from NDGA's two catechol rings, which scavenge oxygen free radicals, alongside inhibition of 5-lipoxygenase and activation of the NRF2 transcription factor pathway. Heron and Yarnell (2001) confirmed that topical application provides localized anti-inflammatory effects while minimizing systemic exposure and associated hepatotoxicity concerns that limit internal use.

\textit{Field preparation}: Identify creosote by its distinctive resinous odor, small bilobed leaves with a waxy coating, and yellow five-petaled flowers. The resin becomes palpably sticky when leaves are crushed between fingers---this tactile assessment indicates high phenolic content. Traditional external preparations involved boiling leaves to create concentrated washes applied to affected joints, or heating branches to create aromatic smoke directed at painful areas. The resinous coating that waterproofs the leaf against desiccation concentrates the bioactive lignans.

\subsection{The Willow Precedent}

Multiple Great Basin tribes utilized willow bark (\textit{Salix} spp.) for pain and fever, a practice that provided the chemical foundation for modern aspirin. Spring-harvested bark contains salicin concentrations up to 12.5\%, compared to 0.08\% in autumn (Shara and Stohs, 2015)---a seasonal variation that traditional harvesters recognized empirically through efficacy differences.

Pharmacokinetic studies confirm that salicin undergoes conversion to saligenin in the small intestine via bacterial glycosidase activity, followed by hepatic oxidation to salicylic acid (Schmid et al., 2001). This metabolic pathway explains why willow bark preparations require 2--3 hours to achieve peak effect, unlike synthetic aspirin's rapid absorption. The conversion also means that fresh willow bark tea delivers a sustained-release analgesic profile that some clinical studies suggest produces fewer gastrointestinal side effects than equivalent aspirin doses.

\textit{Field preparation}: Select young branches approximately thumb-thickness from trees growing near water sources. Using a knife, score the outer bark longitudinally and peel strips of inner bark (cambium layer), which appears pale green to cream-colored. Traditional decoction requires simmering 1--2 teaspoons of dried, crumbled bark per cup of water for 10--15 minutes---a more vigorous extraction than simple steeping, necessary to release compounds from fibrous plant material.

%==============================================================
\section{Sacred Plants and Ceremonial Practice}
%==============================================================

Among the Kawaiisu, cosmological narratives established that four medicines were given to the people at the Beginning: tobacco, nettles, red ants, and jimsonweed. This origin framework reveals the Indigenous understanding that analgesia and visionary experience operated through related mechanisms---both represented encounters with sacred power channeled through beings who had offered themselves for human benefit.

\subsection{The Toloache Complex}

Jimsonweed (\textit{Datura wrightii}), known as \textit{toloache} throughout Southern California, occupied exceptional ceremonial significance. Tubatulabal oral tradition held that the plant had once been a man who transformed himself to help the people ``cure sickness and find spiritual power.'' Unlike other medicinal plants, jimsonweed possessed ontological status as an intentional being.

Contemporary alkaloid analysis illuminates the pharmacological basis of these traditions. \textit{Datura} species contain tropane alkaloids---primarily scopolamine and atropine---that act as competitive antagonists at muscarinic acetylcholine receptors, producing anticholinergic effects including mydriasis, tachycardia, hyperthermia, and delirium with vivid hallucinations (Kohnen-Johannsen and Kayser, 2019). Critically, alkaloid concentrations vary dramatically: young plants exhibit scopolamine-to-atropine ratios of approximately 3:1, which reverses after flowering as scopolamine decreases with plant maturity. Root alkaloid content peaks during winter dormancy---precisely when Kawaiisu tradition restricted toloache use. An individual seed contains approximately 0.1 mg atropine, with fatal adult doses estimated at $>$10 mg atropine or $>$2--4 mg scopolamine.

The preparation protocols documented by Kroeber (1925) and others reflect sophisticated risk management: dried roots were pounded in specially painted mortars reserved exclusively for this purpose and kept in sacred hiding places. The powdered root mixed with hot water was consumed directly from the mortar, with each initiate kneeling while the ceremonial manager supported his head. This single ingestion---taken only once in a lifetime---occurred during male puberty rites, with the visions received during subsequent days of fasting providing lifelong spiritual guidance.

Archaeological evidence from Kern County includes a painted cave featuring a red pinwheel motif interpreted as depicting a \textit{Datura} flower during anthesis (Figure~\ref{fig:pinwheel}). Researchers recovered dozens of masticated plant fiber wads pressed into rock crevices---direct material evidence of consumption at the site. Robinson et al. (2020) used radiocarbon dating and alkaloid residue analysis to confirm ceremonial use spanning multiple centuries.

\begin{figure}[h]
\centering
\includegraphics[width=0.95\columnwidth]{pinwheel-02}
\caption{Pinwheel Cave rock art and \textit{Datura wrightii} flower morphology. The red pinwheel motif depicts the characteristic five-armed unfurling of \textit{Datura} corolla during anthesis. From Robinson et al. (2020) \textit{PNAS} 117:31207. CC BY-NC-ND 4.0.}
\label{fig:pinwheel}
\end{figure}

\textbf{This plant is extremely dangerous and should never be handled or consumed. Observation only.}

\subsection{The Pine Nut Ceremonial Cycle}

For peoples who referred to themselves as ``pine nut eaters,'' the pinyon harvest constituted the central annual ceremonial event. Pine nut gathering defined the great social time of the year, concentrating dispersed family groups in sacred lowland pinyon forests understood as consecrated ground.

Nutritional analysis supports this cultural centrality. Lanner (1981) documented that singleleaf pinyon (\textit{Pinus monophylla}) contains 54\% carbohydrates, 9.5\% protein, and 27\% lipids, with unsaturated fatty acids (oleic, linoleic, linolenic) comprising approximately 85\% of total fatty acids. All twenty amino acids are present. The high carbohydrate content distinguishes \textit{P. monophylla} from other edible pines and explains its role as winter subsistence: caloric density combined with storage stability made pine nuts irreplaceable.

Traditional processing maximized nutritional extraction while creating social cohesion. Green cones were roasted in fire pits to open scales and facilitate seed release---a process that simultaneously reduced resin content and enhanced digestibility. Seeds were then winnowed in basketry trays whose designs, unchanged across generations, encoded cultural continuity in material form.

\textit{Field observation}: Singleleaf pinyon is identifiable by its single needle per fascicle (unique among pines), blue-green coloration, and rounded crown form. Mature cones appear in August--September, with viable seed years occurring at 3--7 year intervals depending on precipitation. The resinous aroma of pinyon smoke remains distinctive and was considered purifying in ceremonial contexts.

\subsection{Puha and Plant Ontology}

Northern Paiute metaphysics understood \textit{puha} (power) to reside throughout the natural world---in animals, plants, rocks, water, and celestial bodies. This power could be acquired through dreams, visions, or rituals, enabling individuals to heal illness, influence weather, or locate lost objects. Plants were beings possessing their own \textit{puha}, deserving of respectful relationship.

Sagebrush burning for ritual purification operates within this framework: the aromatic smoke carries the plant's \textit{puha} into sacred space. The neurological basis for aromatic effects on consciousness is now partially understood. Olfactory receptors project directly to limbic structures including the amygdala and hippocampus without thalamic relay, providing rapid modulation of emotional and memory-related processing (Herz, 2009). Monoterpenes including camphor cross the blood-brain barrier and modulate GABAergic and glutamatergic transmission (Moss et al., 2003)---effects that may underlie the calming or alerting properties attributed to ceremonial aromatics across cultures.

%==============================================================
\section{Material Culture}
%==============================================================

\subsection{Basketry as Technical Practice}

Great Basin women's basketry ranks among the finest textile production in world ethnographic record. Mono Lake, Owens Valley, and Southern Paiute traditions employed distinct material combinations: northern baskets incorporated redbud and bracken fern root for red and black design elements with sedge or willow providing the light ground, while southern work favored devil's claw for black patterning.

The preparation of willow demanded extensive environmental knowledge. Traditionally, branches were gathered in fall and left outside through winter, where cold air dried them sufficiently to permit splitting. Contemporary basketmaker Leah Brady reports that climate change has disrupted this practice: warmer, moister winters prevent adequate air-drying, requiring freezer storage before processing. The intimate oral contact required during willow preparation---holding strands in the mouth while splitting---means basketmakers must now travel considerable distances to locate gathering sites free from pesticide contamination.

\textit{Field practice}: Identify willow species along watercourses by their narrow, lance-shaped leaves and flexible young stems. Traditional selection favored straight, unbranched shoots of the current year's growth, approximately pencil-thickness. To assess flexibility, bend a candidate shoot into a tight arc---suitable material returns to straight without cracking or retaining curvature. Bark color varies by species and affects the finished basket's appearance.

\subsection{Acorn Processing Chemistry}

Western Mono (Monache) relied heavily on acorns, processed through leaching to remove bitter tannins. These polyphenolic compounds bind and precipitate proteins---the same chemistry exploited in leather tanning---and interfere with nutrient absorption if consumed in quantity.

Indigenous California methods varied by tribe but shared core principles now explicable through food chemistry. The Karuk dried acorns for a year, cracked shells with stones, pounded meal using bedrock mortars, placed meal in sand basins, and leached with water allowing tannins to drain into the sand. The Ohlone employed similar mortars visible today throughout Bay Area parklands.

Temperature critically affects the final product. Processing below 150\textdegree F preserves a starch fraction that functions analogously to gluten, allowing cold-leached flour to bind in baked goods without additional ingredients. Hot leaching (boiling) denatures this starch, producing meal requiring egg or other binders. Red oak acorns contain higher tannin concentrations than white oaks, requiring 10--14 days of cold leaching versus 4--7 days for white oak species (Bainbridge, 1986).

\textit{Field observation}: Identify valley oak (\textit{Quercus lobata}) by deeply lobed leaves and elongated acorns, or interior live oak (\textit{Q. wislizeni}) by holly-like evergreen leaves. Examine bedrock mortars where present---the smooth, cup-shaped depressions in granite were created through generations of grinding (Figure~\ref{fig:mortar}). The depth and polish of these mortars indicate intensive, long-term use at specific processing sites.

\begin{figure}[h]
\centering
\includegraphics[width=0.95\columnwidth]{grinding-1}
\caption{Indian Grinding Rock State Historic Park (Chaw'se) contains 1,185 bedrock mortar cups---the largest concentration in North America. CA State Parks.}
\label{fig:mortar}
\end{figure}

\subsection{The Owens Valley Irrigation System}

The Eastern Mono (Owens Valley Paiute) developed what Steward (1930) characterized as ``perhaps the best instance in North America of a group that developed its own system of vegeculture''---irrigation agriculture applied to wild seed-bearing plants rather than domesticated crops. Over sixty miles of hand-dug canals redistributed Sierra Nevada streamflow across the valley floor.

The system required sophisticated governance. Each spring, communities elected a \textit{tuvaij\"{u}} (head irrigator) who assessed land conditions and determined water distribution. Approximately twenty-five men constructed seasonal dams from boulders, brush, and mud, while the \textit{tuvaij\"{u}} directed water into ditch networks using the \textit{pavado}---an eight-foot hardwood digging stick. Northern and southern plots alternated annually, either conserving soil fertility or permitting natural reseeding.

Wilke and Lawton (1976) matched surveyor A.W. Von Schmidt's 1856 maps to Steward's 1933 ethnographic documentation, confirming that Paiute irrigation predated white settlement. George Collins, an Owens Valley Paiute, explained tribal identity through this water relationship: ``We are water ditch coyote children,'' referencing oral traditions in which Coyote placed the people near water in Owens Valley. The ancestral term for irrigation, \textit{tuvadut}, embeds water management at the linguistic core of cultural identity.

\textit{Field observation}: Remnants of these ancient ditches---some miles long and as wide as modern canals---remain visible in the Owens Valley landscape. Examine vegetation patterns for linear anomalies suggesting former water distribution, and note how settlement patterns cluster near former ditch alignments.

\vspace{6pt}
\noindent\rule{\columnwidth}{0.4pt}
\vspace{3pt}

{\scriptsize\textbf{References}: Adams, R.P. (2012) \textit{J. Essent. Oil Res.} 24:341; Bainbridge, D.A. (1986) \textit{Agroforest. Syst.} 4:249; Herz, R.S. (2009) \textit{Int. J. Psychophysiol.} 47:185; Kohnen-Johannsen, K.L. \& Kayser, O. (2019) \textit{Molecules} 24:4238; Kroeber, A.L. (1925) \textit{Handbook of the Indians of California}; Lanner, R.M. (1981) \textit{The Pi\~{n}on Pine}; Moerman, D. (1998) \textit{Native American Ethnobotany}; Moss, M. et al. (2003) \textit{Int. J. Neurosci.} 113:15; Rahman, S. et al. (2011) \textit{eCAM} nep076; Robinson, D.W. et al. (2020) \textit{PNAS} 117:31207; Schmid, B. et al. (2001) \textit{Phytother. Res.} 15:344; Shara, M. \& Stohs, S.J. (2015) \textit{Phytother. Res.} 29:1112; Steward, J.H. (1930) \textit{Papers Michigan Acad.} 12:149; Wilke, P.J. \& Lawton, H.W. (1976) in \textit{Native Californians}; Zigmond, M.L. (1981) \textit{Kawaiisu Ethnobotany}.}

\end{document}
