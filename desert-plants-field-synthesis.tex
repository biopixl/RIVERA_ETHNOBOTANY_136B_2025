\documentclass[10pt,letterpaper,twocolumn]{article}
\usepackage[utf8]{inputenc}
\usepackage[T1]{fontenc}
\usepackage[margin=0.6in]{geometry}
\usepackage{enumitem}
\usepackage{booktabs}
\usepackage{xcolor}
\usepackage{tcolorbox}
\usepackage{titlesec}
\usepackage{multicol}
\usepackage{array}

% Compact spacing
\setlength{\parskip}{3pt}
\setlength{\parindent}{0pt}
\titlespacing*{\section}{0pt}{8pt}{4pt}
\titlespacing*{\subsection}{0pt}{6pt}{2pt}
\setlist{noitemsep,topsep=2pt,leftmargin=*}

% Colors
\definecolor{healgreen}{RGB}{34,139,34}
\definecolor{ritualpurple}{RGB}{102,51,153}
\definecolor{craftbrown}{RGB}{139,90,43}
\definecolor{promptblue}{RGB}{0,90,156}

% Compact boxes
\newtcolorbox{promptbox}{colback=promptblue!10,colframe=promptblue,boxrule=0.5pt,left=3pt,right=3pt,top=2pt,bottom=2pt}

\begin{document}

\begin{center}
{\Large\bfseries Desert Plant Healing, Rituals \& Material Culture}\\[2pt]
{\small Saline Valley Region --- Ge136 Field Synthesis}
\end{center}

\vspace{-6pt}
\hrule
\vspace{6pt}

{\small\textit{Five Indigenous groups---Kawaiisu, Tubatulabal, Western Shoshone, Eastern Mono, and Northern Paiute---developed sophisticated plant knowledge in this region. Their traditions illuminate relationships between people, plants, and place.}}

%==============================================================
\section*{\textcolor{healgreen}{Healing Plants}}
%==============================================================

\textbf{Philosophy}: ``Traditional Shoshone see their food as medicine. They understand that nature truly heals.'' Plants served dual roles as nutrition and pharmacy.

\begin{center}
\small
\begin{tabular}{@{}p{1.8cm}p{4.8cm}@{}}
\toprule
\textbf{Plant} & \textbf{Healing Uses} \\
\midrule
\textbf{Sagebrush} & Leaf tea expels lung/stomach phlegm; induces sweating for colds and fever \\
\textbf{Yerba Santa} & Kawaiisu drank tea daily for one month to treat infections \\
\textbf{Creosote Bush} & Antiseptic washes; arthritis relief; respiratory ailments \\
\textbf{Brittlebush} & Leaf decoction as wash for rheumatic pain (Kawaiisu) \\
\textbf{Plantain} & Leaf poultice for wounds, especially slow-healing burns \\
\textbf{Lomatium} & Root decoction as wash for broken limbs \\
\textbf{Juniper} & Berries eaten fresh or dried; multiple medicinal uses \\
\bottomrule
\end{tabular}
\end{center}

\textbf{Poultice Tradition}: Shoshone and Gosiute prepared plant poultices to treat swelling. Root infusions served as eye medicine. These practices required intimate knowledge of plant parts, seasons, and preparation methods.

\begin{promptbox}
\textbf{Field Observation}: Locate a creosote bush. Crush a leaf and smell the resin. Early peoples recognized this antiseptic property. What does the smell remind you of?
\end{promptbox}

%==============================================================
\section*{\textcolor{ritualpurple}{Sacred Plants \& Rituals}}
%==============================================================

\subsection*{The Four Medicines (Kawaiisu)}

In Kawaiisu cosmology, four medicines were given to the people at the Beginning:

\begin{enumerate}
\item \textbf{Tobacco} (\textit{Nicotiana bigelovii})
\item \textbf{Nettles}
\item \textbf{Red ants}
\item \textbf{Jimsonweed} (\textit{Datura wrightii})
\end{enumerate}

All four induced dreams and visions while also alleviating pain---the boundary between healing and spiritual practice was fluid.

\subsection*{Jimsonweed (Datura)}

\textbf{WARNING}: Datura is highly toxic. Never touch or ingest.

Among the Tubatulabal, jimsonweed ``had once been a man who turned himself into the plant to help the people cure sickness and find spiritual power.''

\textbf{Uses}:
\begin{itemize}
\item \textbf{Vision quests}: Root infusion drunk to obtain visions foretelling the future (winter only---root considered too dangerous other seasons)
\item \textbf{Male puberty rites}: Administered by knowledgeable elders
\item \textbf{Healing}: Treatment for broken bones, arthritis, pain
\item \textbf{Shamanic practice}: Both Kawaiisu and Tubatulabal
\end{itemize}

\textbf{Archaeological evidence}: In a Kern County cave, researchers found Datura-flower rock art and dozens of chewed plant fiber wads pressed into crevices---evidence of ritual use.

\subsection*{Pine Nut Ceremony}

For all Great Basin peoples, pine nut harvest required ceremony:

\begin{quote}
\small\textit{``Gathering is always preceded by ceremony expressing gratitude, performed with drum, song, and dance. Women dance displaying harvest tools---roasting and winnowing baskets whose designs have remained unchanged since antiquity.''}
\end{quote}

The Walker River Paiute hold an annual Pine Nut Festival every September with blessing, dance, and powwow. Pine nuts were winter subsistence; the ceremony honors ancestors and affirms cultural continuity.

\begin{promptbox}
\textbf{Discussion}: Why might vision-seeking plants require elder supervision? What does this suggest about the relationship between individual experience and community knowledge?
\end{promptbox}

\subsection*{Spiritual Power: Puha/Pooha}

Northern Paiute believe power (\textit{puha}) resides in natural objects---animals, plants, rocks, water. This power could be acquired through dreams, visions, or rituals. Plants weren't merely resources but beings with agency, deserving respect.

%==============================================================
\section*{\textcolor{craftbrown}{Material Culture}}
%==============================================================

\subsection*{Basketry}

Mono and Paiute women's basketry ranks among the finest in the world. Numu stories teach basket-makers to ``weave only good thoughts into baskets.''

\begin{center}
\small
\begin{tabular}{@{}p{2.2cm}p{4.4cm}@{}}
\toprule
\textbf{Material} & \textbf{Use \& Source} \\
\midrule
Willow & Primary structure; split branches \\
Yucca root & Structure; red dye for patterns \\
Devil's claw & Black design elements \\
Tule/Cattail & Gathering bags, mats, decoys \\
\bottomrule
\end{tabular}
\end{center}

\textbf{Functions}: Carrying food, sifting grain, cooking (watertight baskets with hot stones), serving, ceremony.

\subsection*{Construction \& Tools}

\begin{itemize}
\item \textbf{Cattail}: Leaves for thatching roofs/walls; flower down for bedding
\item \textbf{Willow}: House poles, sweathouse frames, bows
\item \textbf{Juniper}: Bark for thatch; wood for tools and bows
\item \textbf{Buckbrush}: Straight twigs for arrow foreshafts; firewood
\item \textbf{Elderberry}: Hollowed stalks for flutes
\end{itemize}

\subsection*{Owens Valley Irrigation}

The Eastern Mono developed North America's most sophisticated indigenous irrigation:

\begin{itemize}
\item 60+ miles of canals and ditches
\item Annual election of \textit{tuvaij\"u} (head irrigator)
\item Hand-dug with \textit{pavodo} (8-ft hardwood pole)
\item Rotated irrigated areas to allow recovery
\item Irrigated wild seed-bearing plants (vegeculture)
\end{itemize}

This represents ``perhaps the best instance in North America of a group that developed its own system of vegeculture.''

\begin{promptbox}
\textbf{Landscape Reading}: As you look across the valley, imagine 60 miles of hand-dug canals. What does this infrastructure tell us about social organization, labor, and long-term planning?
\end{promptbox}

%==============================================================
\section*{Plants You May See Today}
%==============================================================

\begin{center}
\small
\begin{tabular}{@{}p{2cm}p{1.2cm}p{3.2cm}@{}}
\toprule
\textbf{Plant} & \textbf{Zone} & \textbf{Traditional Use} \\
\midrule
Creosote bush & Valley & Medicine, fuel \\
Sagebrush & All & Medicine, ceremony \\
Joshua tree & Mid & Fiber, food \\
Pinyon pine & Upper & Nuts (primary food) \\
Juniper & Upper & Food, medicine, tools \\
Saltbush & Valley & Food (greens) \\
Cattail & Springs & Construction, food \\
\bottomrule
\end{tabular}
\end{center}

%==============================================================
\section*{Interactive Learning Prompts}
%==============================================================

\textbf{1. Observation}: Find three plants. For each, hypothesize one possible use based on its properties (texture, smell, structure). Compare to actual uses.

\textbf{2. Ecology \& Culture}: How does elevation shape both plant communities and human subsistence strategies? Trace the seasonal round from valley to mountain.

\textbf{3. Knowledge Systems}: Indigenous peoples knew ``which crops ripened when'' across diverse habitats. What kind of multigenerational observation would this require?

\textbf{4. Material Choices}: Why might willow be preferred for baskets while juniper was chosen for bows? Consider flexibility, strength, and availability.

\textbf{5. Sacred Landscapes}: Paiute call Owens Valley \textit{Payah\"u\"unad\"u} (``land of flowing water''). How does naming reveal relationship to place?

\vspace{6pt}
\hrule
\vspace{4pt}
{\scriptsize\textbf{Sources}: Zigmond (1981) \textit{Kawaiisu Ethnobotany}; Voegelin (1938) \textit{Tubatulabal Ethnography}; Rhode (2002) \textit{Native Plants of Southern Nevada}; Steward (1930) on Owens Valley irrigation; Native Memory Project; Owens Valley Indian Water Commission.}

\end{document}
