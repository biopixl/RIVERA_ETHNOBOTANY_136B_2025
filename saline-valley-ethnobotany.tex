\documentclass[12pt,letterpaper]{article}
\usepackage[utf8]{inputenc}
\usepackage[T1]{fontenc}
\usepackage{geometry}
\usepackage{natbib}
\usepackage{graphicx}
\usepackage{hyperref}
\usepackage{booktabs}
\usepackage{longtable}
\usepackage{array}
\usepackage{enumitem}
\usepackage{float}

\geometry{margin=1in}

\graphicspath{{figures/extracted/}{figures/maps/}}

\title{Ethnobotany of the Saline Valley Region:\\
Plant Uses and Rituals Among Indigenous Peoples of the\\
Southern Great Basin and Eastern Sierra Nevada}

\author{Research Report for Ge136}

\date{\today}

\begin{document}

\maketitle

\begin{abstract}
This report documents traditional plant uses and rituals among five Indigenous groups with historical connections to the Saline Valley region of California: the Kawaiisu, Tubatulabal, Newe Sogobia (Western Shoshone), N\"u\"um\"u Wit\"u (Eastern Mono/Monache), and N\"u\"um\"u (Northern Paiute). Drawing on ethnographic literature and contemporary sources, this synthesis examines food plants, medicinal applications, ceremonial uses, and material culture across these interconnected Great Basin and Sierra Nevada peoples.
\end{abstract}

\tableofcontents
\newpage

%-----------------------------------------------------------------------
\section{Introduction}
%-----------------------------------------------------------------------

The Saline Valley, located in the northern Mojave Desert of eastern California, sits at a cultural crossroads between Great Basin and Sierra Nevada peoples. For millennia, Indigenous groups traversed this region, utilizing its diverse plant communities from hot desert scrublands to cold mountain forests. The Timbisha Shoshone maintained direct connections to Saline Valley, using its salt deposits as a trade commodity \citep{factcards2024}. Neighboring groups---the Kawaiisu to the south, Tubatulabal to the west, Eastern Mono to the northwest, and Northern Paiute to the north---shared linguistic ties (Numic language family) and ecological knowledge systems.

This report synthesizes ethnobotanical knowledge for five groups with cultural presence in or near the Saline Valley region:

\begin{enumerate}
    \item \textbf{Kawaiisu} --- Tehachapi Mountains and southern Sierra Nevada
    \item \textbf{Tubatulabal} --- Upper Kern River Valley
    \item \textbf{Newe Sogobia (Western Shoshone)} --- Great Basin, including Death Valley and Saline Valley
    \item \textbf{N\"u\"um\"u Wit\"u (Eastern Mono/Monache)} --- Owens Valley and eastern Sierra Nevada
    \item \textbf{N\"u\"um\"u (Northern Paiute)} --- Owens Valley to northern Nevada/Oregon
\end{enumerate}

All five groups developed sophisticated knowledge of native plants, understanding ``which plants and plant parts were useful for curing certain ailments, which produced colorful dyes, which would keep spirits away, and `which crops ripened when' in a particular locality'' \citep{rhode2002}.

\begin{figure}[H]
\centering
\includegraphics[width=0.95\textwidth]{saline_valley_territories_geo.png}
\caption{Indigenous territories of the Saline Valley region overlaid on a shaded relief map showing Basin and Range geological structure. The characteristic north-south trending mountain ranges (Sierra Nevada, Inyo Mountains, Panamint Range) separated by down-dropped valleys (grabens) created distinct ecological zones that influenced Indigenous settlement and subsistence patterns. Major faults (dashed lines) mark range-front boundaries. Boundaries represent approximate traditional territories based on ethnographic sources including \citet{kroeber1925}, \citet{steward1933}, \citet{steward1938}, and \citet{zigmond1981}. Saline Valley occupied a transitional zone where multiple groups converged for salt procurement and seasonal resource exploitation.}
\label{fig:territories}
\end{figure}

%-----------------------------------------------------------------------
\section{Cultural Anthropology of the Saline Valley Region}
%-----------------------------------------------------------------------

Before examining the ethnobotanical knowledge of individual groups, this section provides anthropological context for understanding the cultural systems within which plant knowledge was embedded. The peoples of the Saline Valley region developed sophisticated social organizations, maintained extensive trade networks, and elaborated rich cosmological traditions that gave meaning to their relationships with the natural world.

\subsection{The Numic Expansion and Linguistic Connections}

The Indigenous peoples of the Saline Valley region share deep linguistic and cultural connections through the Numic branch of the Uto-Aztecan language family. The Kawaiisu, Western Shoshone, Eastern Mono, and Northern Paiute all speak Numic languages, while the Tubatulabal speak a distinct Uto-Aztecan language that represents an earlier stratum of occupation in the southern Sierra Nevada.

Archaeological and linguistic evidence suggests that Numic-speaking peoples expanded across the Great Basin approximately 1,000 years ago, a phenomenon known as the ``Numic Spread.'' This expansion brought with it shared mythological traditions, social organizational principles, and subsistence strategies that created a cultural continuum across the region. The Kawaiisu represent the southernmost extension of this Numic cultural sphere, while maintaining trading relationships with non-Numic groups including the Yokuts and Tubatulabal.

\subsection{Social Organization and Political Structure}

\subsubsection{Band and District Organization}

Julian Steward's foundational ethnographic work documented the political organization of Great Basin peoples, distinguishing between ``family'' groups in resource-poor areas and ``band'' groups in areas of greater resource concentration. The Owens Valley Paiute represented the preeminent example of band organization, as the valley's abundant resources supported larger, more permanent settlements with defined territorial boundaries.

Owens Valley was divided into districts, each possessing ``communistic hunting and seed rights, political unity, and a number of villages.'' These districts included Round Valley (\textit{kwina pat\"u}, ``north place'') and Bishop (\textit{pitana pat\"u}, ``south place''). Deep Springs Valley was called \textit{ozanwit\"u}, meaning ``salt place,'' reflecting the importance of saline resources in regional identity. Saline Valley itself served as a crucial salt-procurement zone shared among neighboring groups.

\subsubsection{Leadership and Governance}

Hereditary chiefs led communal activities among the Owens Valley Paiute, though their authority was advisory rather than coercive. The position of \textit{tuvaij\"u} (head irrigator) was elected annually, representing a specialized leadership role tied to the sophisticated irrigation system. The \textit{tuvaij\"u} assessed land conditions each spring and directed water distribution, coordinating the labor of approximately twenty-five men who constructed seasonal dams from boulders, brush, and mud.

Among the Kawaiisu and other groups with more dispersed populations, leadership was even more diffuse. Shamans (\textit{puagants}) wielded considerable influence through their ability to manipulate \textit{puha} (spiritual power), but political authority remained fundamentally egalitarian.

\subsection{Seasonal Rounds and Settlement Patterns}

\subsubsection{The Timbisha Shoshone Model}

The Timbisha Shoshone, who maintained direct connections to Saline Valley, exemplify the seasonal migration patterns characteristic of the region. Both oral traditions and archaeological evidence confirm that traditional Timbisha people moved seasonally between semi-permanent villages throughout greater Death Valley.

During summer months, when valley floor temperatures became unbearable, family groups migrated to cooler mountain elevations. Seasonal camps at Wildrose Canyon and Hunter Mountain served as communal gathering sites where fifty to one hundred individuals harvested pi\~{n}on nuts, seeds, roots, and berries. Hunting focused on bighorn sheep, mule deer, yellow-bellied marmot, and black-tailed jackrabbit. These summer communes also served critical social functions: relatives reacquainted themselves, marriages were arranged, and ceremonial knowledge was transmitted across generations.

With the onset of cooler weather, groups descended to valley floor villages where mesquite groves and springs provided winter resources. Julian Steward documented population distributions of approximately 65 persons in Saline Valley, 150--160 in Little Lake and the Coso Range, 100 in northern Panamint Valley, and 42 in northern Death Valley.

\subsubsection{The Owens Valley Pattern}

The Eastern Mono (Owens Valley Paiute) developed a more sedentary pattern enabled by their irrigation agriculture. Villages along Owens River and its tributaries---including settlements at what is now Manzanar---supported year-round occupation, though seasonal movements to pine nut groves and hunting grounds continued. Archaeological evidence from the Manzanar area reveals Paiute villages along Bairs, Georges, Shepherds, and Symmes creeks, with occupation spanning at least 1,500 years.

\subsection{Trade Networks and Inter-Tribal Relations}

\subsubsection{Salt as Currency}

Saline Valley's salt deposits positioned it at the center of regional exchange networks. The Timbisha Shoshone extracted salt for trade with neighboring groups, including the Owens Valley Paiute, who in turn traded brown-ware pottery for salt. Archaeological evidence from the Manzanar area confirms the exchange of salt from Saline Valley for pottery and other goods.

\subsubsection{Trans-Sierra Exchange}

Trade routes crossed the Sierra Nevada, connecting Great Basin peoples with California groups. The Kawaiisu maintained trading relationships with the Yokuts and Tubatulabal to the north, though they traded less with the Chemehuevi in the desert to the east. Rabbit skin blankets, produced by Great Basin groups with abundant hare and cottontail populations, were traded to the Central and Southern Miwok. The Tubatulabal served as intermediaries, receiving blankets from some groups while trading them to others.

\subsubsection{Obsidian and Shell Bead Networks}

Long-distance exchange networks brought obsidian from volcanic sources in the Cascade Range and eastern Sierra Nevada to communities throughout the region. Chemical analyses of artifacts reveal increasing geochemical diversity in obsidian and increasing density of non-local shell beads through time, indicating expanding trade connections.

Olivella shell beads, manufactured by Chumash peoples on the Santa Barbara Channel, served as currency throughout California and the Great Basin. Bennyhoff and Hughes documented extensive ``Shell Bead and Ornament Exchange Networks between California and the Western Great Basin,'' with chemical analyses suggesting most beads found in Owens Valley originated in southern California.

\subsubsection{Kinship and Festival Networks}

Trade relationships were reinforced through kinship ties. The Paiutes commonly attended festivals hosted by neighboring tribes, ``largely spurred through kinship ties.'' Intermarriage between the Owens Valley Paiute and Western Mono was particularly common, as the two groups shared cultural and linguistic connections.

\subsection{Cosmology and Oral Traditions}

\subsubsection{The Numic Creation Narrative}

Numic peoples share a foundational creation narrative involving Ocean Woman and the trickster figure Coyote. According to this tradition, Ocean Woman created the earth with the assistance of other Immortals. She then placed the ancestors of the Paiute and Shoshone peoples in a sack, basket, or water jug, entrusting it to Coyote for transport.

Coyote, ever curious and irresponsible, opened the container before reaching his destination. The people escaped and scattered in all directions---explaining why different groups now occupy different territories. ``He grew tired and set the basket on the ground. The children came out of the basket and ran away. They scattered in every direction. Coyote tried to catch them, but he couldn't. That is why people are all over the earth.''

\subsubsection{Wolf and Coyote: Death and Mortality}

A complementary narrative explains the origin of death. Wolf declared that people could be brought back to life after they died. Coyote argued that if people returned from death, there would soon be too many of them. Wolf agreed that Coyote was right---but then arranged for Coyote's own son to be the first to die. When Coyote asked Wolf to bring his son back to life, Wolf reminded Coyote that he had insisted on death, and so his son must remain dead. This narrative establishes Coyote as responsible for human mortality while also demonstrating the consequences of his own thoughtlessness.

\subsubsection{The Kawaiisu Coyote Cycle}

Among the Kawaiisu, the ``Coyote Cycle'' comprises the primary body of oral literature. Maurice Zigmond documented these narratives over nearly four decades of research. Coyote appears as the most human of all the animals---simultaneously a culture hero and a fool whose choices often bring suffering. Mountain Lion, by contrast, ``thinks things through and makes the right choices.'' The Kawaiisu say that Mountain Lion taught them the right way, but they chose to follow Coyote.

\subsubsection{Yahwe'era: The Supernatural Gamekeeper}

Among the Kawaiisu, \textit{Yahwe'era} serves as the master of game animals---a dominant religious figure among aboriginal hunting peoples. This being controls access to prey species and must be propitiated through proper ritual behavior. Disrespectful treatment of animal remains or excessive killing can cause \textit{Yahwe'era} to withhold game, leading to starvation.

\subsubsection{Puha/Pooha: Distributed Spiritual Power}

Northern Paiute metaphysics understood \textit{puha} (or \textit{pooha}) as a pervasive supernatural power residing in natural objects: the sun, moon, stars, thunder, wind, water, rocks, and plants all possessed \textit{puha}. This power could be acquired through dreams, visions, or rituals, enabling individuals---particularly shamans---to heal illness, influence weather, or locate lost objects. Plants were not inert resources but beings possessing their own \textit{puha}, deserving of respectful relationship.

\subsection{The Ghost Dance Movement}

The Ghost Dance represents the most significant pan-Indian religious movement to emerge from the Great Basin. In 1889, a Northern Paiute man named Wovoka, born near Walker Lake in western Nevada, experienced a prophetic vision during a solar eclipse. He saw the resurrection of Native dead and the removal of white settlers from North America. To bring this vision to pass, Wovoka taught that Native peoples must live righteously and perform a traditional round dance.

The movement spread rapidly across the Great Basin and Plains. Emissaries from the Arapaho, Cheyenne, Ute, Shoshone, and Sioux traveled to Nevada to learn from the Paiute prophet. Though the movement's association with the Wounded Knee massacre of 1890 led to suppression, the Ghost Dance did not disappear but went underground. Wovoka continued to spread its message until his death in 1932 on the Walker River Indian Reservation.

\subsection{Historical Contact and Colonial Impacts}

\subsubsection{Early Contact and the Owens Valley War}

The first sustained contact between Euro-Americans and Owens Valley peoples occurred in the mid-nineteenth century. In 1861, Samuel Addison Bishop and other ranchers began raising cattle on the luxuriant grasses that Paiute irrigation had maintained for centuries. Conflict over land and water use escalated rapidly.

A harsh winter in 1861--1862 forced Paiute and settlers into open warfare. In 1863, the U.S. Cavalry and a group of settlers drove more than thirty N\"u\"um\"u into Owens Lake, then shot them as they tried to swim to safety. Later that year, the military forcibly marched nearly 1,000 N\"u\"um\"u from Payah\"u\"unad\"u to Fort Tejon, more than 200 miles to the south. Many tribal members died of thirst or starvation along the way. Elder Irene Button recalled that ``They were forced to walk all the way around the eastern shore of Owens Lake''---a distance of approximately 108 miles.

Many Paiute eventually returned to Owens Valley, where they lived in camps near towns and farms, integrating wage labor with traditional food gathering. By 1866, they had become indispensable to the Owens Valley agricultural economy.

\subsubsection{The Los Angeles Aqueduct and Water Dispossession}

The construction of the Los Angeles Aqueduct (1908--1913) represented a second catastrophic dispossession. Fred Eaton, mayor of Los Angeles, promoted a plan to take Owens Valley water to the growing city via a 233-mile gravity-fed aqueduct. Under the direction of William Mulholland, the city secretly acquired water rights throughout the valley.

Within approximately ten years, the aqueduct had completely drained Owens Lake (\textit{Patsiata}), destroying navigable waters, wildlife habitat, and a crucial food source. The dry lakebed became the largest source of particulate dust pollution in the United States. By the early 1930s, the Los Angeles Department of Water and Power owned nearly all of the valley's farmland and water rights.

During this period, the utility authored a report titled ``The Owens Valley Indian Problem,'' which recommended removing the N\"u\"um\"u from the valley entirely---or, failing that, confining them to reservations. A 1939 land exchange transferred 2,913.5 acres held in trust for Owens Valley Paiute Indians for 1,391.48 acres owned by Los Angeles---crucially, exclusive of all water rights.

\subsubsection{Timbisha Shoshone and Death Valley National Monument}

The establishment of Death Valley National Monument in 1933 devastated Timbisha Shoshone lifeways. All hunting, food gathering, and travel to seasonal camps were banned within monument boundaries. The summer migration that had sustained the people for millennia was used against them: when families moved to mountain camps, park rangers destroyed their valley-floor homes with high-pressure hoses.

Only in 2000, through the Timbisha Shoshone Homeland Act, did the tribe regain formal land rights within Death Valley National Park. The act established a 314-acre village site at Furnace Creek plus special-use areas at Wildrose, Hunter Mountain, and mesquite gathering grounds. Yet climate change now threatens these hard-won lands: as temperatures rise, the valley floor becomes increasingly uninhabitable.

\subsection{Contemporary Tribal Presence and Revitalization}

\subsubsection{Federally Recognized Tribes}

Several federally recognized tribes maintain connections to the Saline Valley region:

\begin{itemize}
    \item Big Pine Paiute Tribe of the Owens Valley
    \item Bishop Paiute Tribe
    \item Lone Pine Paiute-Shoshone Reservation
    \item Fort Independence Indian Reservation
    \item Timbisha Shoshone Tribe
    \item T\"ubatulabal Tribe
\end{itemize}

\subsubsection{The Owens Valley Indian Water Commission}

In 1991, four Owens Valley tribes established the Owens Valley Indian Water Commission to address impacts of Los Angeles' water extractions on reservation environments. The Fort Independence Reservation joined in 1992. The Commission has pursued settlement of water rights through negotiation with the Los Angeles Department of Water and Power, seeking restoration of traditional water access and environmental remediation.

George Collins, an Owens Valley Paiute, explained tribal identity through water: ``We are water ditch coyote children,'' referencing oral traditions in which Coyote placed the people near water in Owens Valley. The ancestral term for irrigation, \textit{tuvadut}, embeds water management at the linguistic core of cultural identity.

\subsubsection{Land Repatriation}

Recent land repatriation efforts have returned ancestral territories to tribal control. In 2023, Western Rivers Conservancy and the T\"ubatulabal Tribe preserved Fay Creek Ranch, a series of freshwater springs feeding the South Fork Kern River. This represents the first land repatriated to the Tubatulabal, allowing tribal members to reconnect with ancestral lands, hold ceremonies, and practice traditional plant gathering.

%-----------------------------------------------------------------------
\section{The Kawaiisu}
%-----------------------------------------------------------------------

\subsection{Cultural Context}

The Kawaiisu traditionally occupied the Tehachapi Mountains and adjacent areas of the southern Sierra Nevada, with cultural connections extending into the western Mojave Desert. Maurice L. Zigmond's \textit{Kawaiisu Ethnobotany} (1981) remains the authoritative source on their plant knowledge \citep{zigmond1981}.

\subsection{Food Plants}

\subsubsection{Acorns and Pine Nuts}

The Kawaiisu gathered acorns from California scrub oak (\textit{Quercus berberidifolia}) and stored them in elevated granaries for long-term preservation. Pi\~{n}on pine nuts (\textit{Pinus monophylla}) served as another dietary staple \citep{ethnoherbalist2024}.

\subsubsection{Seeds and Greens}

\begin{itemize}
    \item \textbf{Chia} (\textit{Salvia columbariae}) --- Seeds parched and ground \citep{zigmond1981}
    \item \textbf{Dock} (\textit{Rumex} spp.) --- Seeds parched, pounded, and mixed into porridge \citep{zigmond1981}
    \item \textbf{Lomatium} (\textit{Lomatium utriculatum}) --- Roots consumed as food \citep{zigmond1981}
\end{itemize}

\subsection{Medicinal Plants}

\begin{longtable}{p{3.5cm}p{4cm}p{6cm}}
\toprule
\textbf{Plant} & \textbf{Scientific Name} & \textbf{Use} \\
\midrule
\endhead
Virgin River Brittlebush & \textit{Encelia virginensis} & Leaf and flower decoction as wash for rheumatic pain; also for horse injuries \\
Yerba Santa & \textit{Eriodictyon californicum} & Tea consumed instead of water for one month to treat gonorrhea \\
Lomatium & \textit{Lomatium utriculatum} & Root decoction as wash for broken limbs \\
California Juniper & \textit{Juniperus californica} & Berries eaten fresh or dried and pulverized \\
White Sage & \textit{Salvia apiana} & Multiple medicinal applications \\
\bottomrule
\caption{Kawaiisu Medicinal Plants}
\end{longtable}

\subsection{Material Culture}

\begin{itemize}
    \item \textbf{Cattail} (\textit{Typha} spp.) --- Leaves for thatching roofs and walls; flower down for bedding
    \item \textbf{Willow} (\textit{Salix} spp.) --- Poles for house and sweathouse construction; branches for bows
    \item \textbf{Buckbrush} (\textit{Ceanothus cuneatus}) --- Firewood; straight twigs for arrow foreshafts
    \item \textbf{California Juniper} --- Shredded bark for roof thatch; wood for kitchen tools and bows
\end{itemize}

\subsection{Ceremonial and Spiritual Uses}

\subsubsection{The Four Medicines}

According to Kawaiisu cosmology, in the Beginning, four medicines were given to the people: tobacco, nettles, red ants, and jimsonweed (\textit{Datura wrightii}). All four were used to induce dreams and visions as well as alleviate pain \citep{vredenburgh2024}.

\subsubsection{Jimsonweed (\textit{Datura wrightii})}

Jimsonweed held central importance in Kawaiisu spiritual practice. An infusion of the roots was drunk to obtain visions that could foretell the future, generally limited to winter when the root was considered less dangerous. The same infusion helped heal broken bones. For arthritis, roots were mashed and soaked, and sore limbs bathed in the preparation \citep{vredenburgh2024}.

Shamanic use of Datura was practiced by the Kawaiisu, and puberty rites for males involving the plant were also conducted \citep{netroots2024}.

\subsubsection{Tobacco}

Eating tobacco (\textit{Nicotiana bigelovii}) was another accepted procedure for gaining beneficial visions \citep{vredenburgh2024}.

%-----------------------------------------------------------------------
\section{The Tubatulabal}
%-----------------------------------------------------------------------

\subsection{Cultural Context}

The T\"ubatulabal have been stewards of the Kern Valley for thousands of years, with their traditional homeland encompassing the upper Kern River drainage in the southern Sierra Nevada. Erminie W. Voegelin's \textit{Tubatulabal Ethnography} (1938) provides the foundational ethnographic documentation \citep{voegelin1938}.

\subsection{Food Plants}

\subsubsection{Primary Staples}

\begin{itemize}
    \item \textbf{Acorns} --- Gathered from the Greenhorn Mountains in late fall, dried in the sun, stored in elevated granaries
    \item \textbf{Pi\~{n}on Nuts} --- Collected from eastern Sierra slopes in early fall, heated to open cones, dried, stored in stone-lined pits
\end{itemize}

\subsubsection{Other Plant Foods}

\begin{itemize}
    \item Seeds: chia, wild oats
    \item Leaves, bulbs, tubers, and roots
    \item Berries: juniper, manzanita, gooseberries, boxthorn
    \item Sugar crystals from honey dew cane stalks
\end{itemize}

Plants were prepared by boiling into mush, roasting, or baking in earth ovens. Berries could be pounded, mixed with water, shaped into cakes, dried, and stored for winter \citep{factcards2024tubatulabal}.

\subsection{Material Culture}

\subsubsection{Basketry}

Baskets were made from:
\begin{itemize}
    \item \textbf{Split willow branches} --- Primary structural material
    \item \textbf{Yucca roots} (\textit{Yucca}) --- Alternative structural material; red-colored pieces for design patterns
    \item \textbf{Devil's claw} (\textit{Proboscidea}) --- Black patterns on coiled baskets
\end{itemize}

Both twining and coiling methods were used, with designs applied only to coiled baskets. Baskets served for carrying food, sifting grains, cooking, and serving \citep{factcards2024tubatulabal}.

Estefana Miranda, a noted Tubatulabal elder, knew how to make ``flat round'' baskets used for both sifting and ceremonies, as well as processing pi\~{n}on nuts from Walker Pass, Kennedy Meadows, and Greenhorn Mountain areas.

\subsubsection{Musical Instruments}

\begin{itemize}
    \item Flutes made from elderberry (\textit{Sambucus}) stalks
    \item Various plant-fiber rattles
    \item Quill whistles
    \item Musical bows
\end{itemize}

\subsubsection{Hunting Equipment}

Many kinds of nets, traps, and snares were woven from plant fibers. Large nets were set up across canyons for communal rabbit drives \citep{factcards2024tubatulabal}.

\subsection{Ceremonial and Medicinal Uses}

\subsubsection{Jimsonweed}

Among the Tubatulabal, jimsonweed was believed to have once been a man who transformed himself into the plant to help the people cure sickness and find spiritual power. According to anthropologist Charles Smith, ``No form of worship was attached to the plant or its use. Rather, the plant's medicinal properties and its use in obtaining supernatural power and longevity were emphasized'' \citep{everyculture2024}.

Sacred Datura was applied topically or taken internally to treat:
\begin{itemize}
    \item Broken bones or sprains
    \item Associated pain and swelling
    \item Intestinal bloat and constipation
\end{itemize}

\subsubsection{Shamanism}

Both men and women could become shamans among the Tubatulabal. Male shamans had both curing and witching powers; female shamans had only witching power. The Tubatulabal believed in numerous spirits, both human and animal, all treated with respect \citep{factcards2024tubatulabal}.

\subsection{Contemporary Revival}

In 2023, Western Rivers Conservancy and the T\"ubatulabal Tribe preserved Fay Creek Ranch, a series of freshwater springs feeding the South Fork Kern River. This represents the first land repatriated to the tribe, allowing tribal members to reconnect with ancestral lands, hold ceremonies, and practice traditional plant gathering \citep{wrc2024}.

%-----------------------------------------------------------------------
\section{Newe Sogobia (Western Shoshone)}
%-----------------------------------------------------------------------

\subsection{Cultural Context}

The Western Shoshone, who call themselves Newe (``The People''), traditionally occupied the vast Great Basin territory they call Newe Sogobia. The Timbisha Shoshone (Death Valley/Panamint Shoshone) maintained direct connections to Saline Valley, Death Valley, and Panamint Valley \citep{factcards2024}.

\subsection{Food Plants}

\subsubsection{Pine Nuts}

Pine nuts took the place of acorns for the Shoshone, who had few oak trees in their territory. Nuts were gathered in early fall, with each family attempting to harvest enough to last through winter \citep{factcards2024}.

\subsubsection{Seeds and Mesquite}

Seeds from many grasses and plants were gathered in summer, roasted to dry, and ground into flour for storage. In the desert, mesquite (\textit{Prosopis}) pods were particularly important. The beans were pounded to make flour for flat cakes \citep{factcards2024}.

\subsubsection{Green Plants}

Fresh green plants were scarce in the desert. Those available (such as clover) required boiling and squeezing to remove bitter mineral salts from desert soil \citep{factcards2024}.

\subsection{Medicinal Plants}

Traditional Shoshone view their food as medicine, understanding that ``nature truly heals'' \citep{nativememory2024}.

\begin{longtable}{p{3.5cm}p{4cm}p{6cm}}
\toprule
\textbf{Plant} & \textbf{Scientific Name} & \textbf{Use} \\
\midrule
\endhead
Sagebrush & \textit{Artemisia tridentata} & Leaf tea to expel lung/stomach phlegm; induces sweating for colds \\
Black Hawthorn & \textit{Crataegus douglasii} & Berries, flowers, or leaves added to food for health \\
Canadian Goldenrod & \textit{Solidago canadensis} & ``Food as medicine'' --- consumed for health benefits \\
Silver Buffaloberry & \textit{Shepherdia argentea} & Cooked berries high in lycopene for cardiovascular health \\
Broadleaf Plantain & \textit{Plantago major} & Leaf poultice for wounds, especially slow-healing burns \\
\bottomrule
\caption{Western Shoshone Medicinal Plants}
\end{longtable}

The Shoshone and Gosiute prepared plant poultices to treat swelling, and root infusions for eye medicine \citep{redbuttegarden2024}.

\subsection{Trade Networks}

The California Shoshone maintained contacts with groups west of the Sierra Nevada, including trade relationships with the Tubatulabal and, through them, the Yokuts of the Central Valley. Salt from Saline Valley served as a valuable trade commodity \citep{factcards2024}.

\subsection{Ceremonial Practices}

\subsubsection{Pine Nut Ceremony}

For the Shoshone, gathering pine nuts is always preceded by ceremony expressing gratitude, performed with drum, song, and dance. Women dance displaying harvest tools---roasting and winnowing baskets whose designs have remained unchanged since antiquity. A traditional blessing is given before harvest begins \citep{nativememory2024harvest}.

The Yomba Shoshone Tribe preserves sacred ceremonies rooted in Great Basin traditions of harmony with land, spirits, and ancestors. These rituals blend vision quests, communal dances, and seasonal observances, emphasizing spiritual power gained through dreams and nature reverence \citep{yomba2024}.

\subsubsection{Musical Instruments}

Some Shoshone people used a four-holed flute made from elderberry wood for festival music \citep{factcards2024}.

%-----------------------------------------------------------------------
\section{N\"u\"um\"u Wit\"u (Eastern Mono/Monache)}
%-----------------------------------------------------------------------

\subsection{Cultural Context}

The Mono (N\"u\"um\"u, meaning ``People'') traditionally inhabit the central Sierra Nevada, Eastern Sierra south of Bridgeport, Mono Basin, and adjacent Great Basin areas. Two main dialect groups exist: Eastern Mono (Owens Valley Paiute) in the Owens River Valley, and Western Mono (Monachi) on the western Sierra slopes \citep{wikipedia2024mono}.

\subsection{Food Plants}

\subsubsection{Eastern Mono (Owens Valley Paiute)}

The Eastern Mono emphasized:
\begin{itemize}
    \item \textbf{Pine nuts} from pi\~{n}on groves --- roasted, shelled, stored or ground into flour
    \item \textbf{Seeds} from grasses and sages --- beaten into baskets
    \item Acorns obtained primarily through trade (scarce in drier eastern territories)
\end{itemize}

\subsubsection{Western Mono (Monache)}

The Western Mono, occupying Sierra Nevada foothills at 3,000--7,000 feet elevation, relied heavily on:
\begin{itemize}
    \item \textbf{Acorns} --- primary dietary staple, processed by leaching tannins through water immersion, ground into meal for mush or cakes
\end{itemize}

The Western Mono inhabited oak woodlands enabling specialized acorn-based adaptations \citep{wikipedia2024mono}.

\subsection{Indigenous Irrigation Agriculture}

The N\"u\"um\"u of Owens Valley developed North America's most sophisticated indigenous irrigation system. For millennia, they spread water across the land to sustain plants and animals, calling their homeland Payah\"u\"unad\"u (``the land of flowing water'') \citep{oviwc2024}.

\subsubsection{Water Management System}

\begin{itemize}
    \item Each year, the people elected a \textbf{tuvaij\"u} (head irrigator) to assess lands and decide where to spread water
    \item Ditches and waterways were dug by hand using the \textbf{pavodo}---an eight-foot hardwood pole
    \item Irrigated areas were alternated to allow land recovery
    \item Over sixty miles of indigenous waterworks irrigated the valley
\end{itemize}

This vegeculture system included irrigation of a variety of seed-bearing plants, representing ``perhaps the best instance in North America of a group that developed its own system of vegeculture'' \citep{steward1930}.

\subsection{Basketry}

Mono women's basketry ranks among the finest in the world. Their extensive plant knowledge remains relevant today because ``native plants are better suited to our environment'' \citep{altapeak2024}.

\subsection{Ceremonial Life}

The Owens Valley Paiute participated in round dances and held annual harvest festivals. Elaborate puberty ceremonies were conducted for girls. Mourning was expressed through ``The Cry,'' a ceremony Yuman in origin that included ritual face washing after a year of mourning \citep{britannica2024}.

%-----------------------------------------------------------------------
\section{N\"u\"um\"u (Northern Paiute)}
%-----------------------------------------------------------------------

\subsection{Cultural Context}

The Northern Paiute (Numu) aboriginal lands span parts of California, Nevada, eastern Oregon, and southern Idaho. The landscape includes flat bottomland, painted desert, canyon, plateau, and mountain areas---all holding enormous spiritual significance \citep{nativeland2024paiute}.

\subsection{Spiritual Worldview: Puha/Pooha}

Northern Paiute believe that power (\textit{pooha} or \textit{puha}) resides in natural objects including animals, plants, and geographical features. Puha was understood as a pervasive supernatural power in the sun, moon, stars, thunder, wind, water, rocks, and plants. It could be acquired through dreams, visions, or rituals, enabling individuals---particularly shamans---to manipulate it for healing or finding lost objects \citep{nevadasindian2024}.

\subsection{Food Plants}

\subsubsection{Seeds and Roots}

Important plant foods included:
\begin{itemize}
    \item \textbf{Seeds}: Indian ricegrass, sunflowers
    \item \textbf{Roots and bulbs}: sego lily, bitterroot, yampa, biscuitroot
    \item \textbf{Seasonal berries}: chokecherries, currants
\end{itemize}

Root harvesting occurred during April--May ``root camps'' lasting approximately six weeks. Over 30 species are documented in ethnobotanical records, with emphasis on knowledge of habitat timing to maximize yields without depleting stands \citep{britannica2024paiute}.

\subsubsection{Preservation}

Air-drying was the primary preservation technique, enabling storage for winter consumption.

\subsection{Material Culture}

\subsubsection{Tule and Cattails}

Numu people traditionally used tule (\textit{Schoenoplectus acutus}) to make:
\begin{itemize}
    \item Gathering bags
    \item Matting
    \item Hunting decoys
    \item Boats (secured with cattails)
\end{itemize}

\subsubsection{Basketry}

A few Numu artists continue traditional basket-making, which requires ``an intimate connection with Numu plants, land, and language.'' Numu stories teach basket-makers to weave only good thoughts into baskets as they create them \citep{nevadasindian2024}.

\subsection{Pine Nut Festival}

Every third weekend of September, several hundred American Indians and visitors gather at the Walker River Paiute Tribe Reservation in Schurz for the Pine Nut Festival. Events include:
\begin{itemize}
    \item Powwow
    \item Hand games
    \item Pine nut blessing and dance
\end{itemize}

During the pine nut dance, dancers move on sacred ground around a pine tree with pi\~{n}on offerings for the blessing. This spiritual ceremony dates back over a hundred years to when pine nuts were winter subsistence for Great Basin Indians. Today, the dance honors ancestors, preserves native traditions, and revives spiritual practices \citep{nativememory2024harvest}.

%-----------------------------------------------------------------------
\section{Comparative Analysis: Shared Plant Knowledge}
%-----------------------------------------------------------------------

\subsection{Common Food Plants Across Groups}

\begin{longtable}{p{3cm}p{2cm}p{2cm}p{2cm}p{2cm}p{2cm}}
\toprule
\textbf{Plant} & \textbf{Kawaiisu} & \textbf{Tubatulabal} & \textbf{W. Shoshone} & \textbf{E. Mono} & \textbf{N. Paiute} \\
\midrule
\endhead
Pi\~{n}on Pine & $\checkmark$ & $\checkmark$ & $\checkmark$ & $\checkmark$ & $\checkmark$ \\
Acorns & $\checkmark$ & $\checkmark$ & Trade & $\checkmark$* & Limited \\
Chia & $\checkmark$ & $\checkmark$ & --- & --- & --- \\
Juniper & $\checkmark$ & $\checkmark$ & $\checkmark$ & --- & --- \\
Mesquite & --- & --- & $\checkmark$ & --- & --- \\
Wild seeds & $\checkmark$ & $\checkmark$ & $\checkmark$ & $\checkmark$ & $\checkmark$ \\
\bottomrule
\caption{Comparative Food Plant Use (*Western Mono primary; Eastern Mono through trade)}
\end{longtable}

\subsection{Shared Ceremonial Elements}

\subsubsection{Pine Nut Ceremonies}

All five groups participated in some form of pine nut harvest ceremony, though intensity varied with ecological availability. These ceremonies persist today as affirmations of cultural heritage.

\subsubsection{Jimsonweed Use}

The Kawaiisu and Tubatulabal both used jimsonweed (\textit{Datura wrightii}) for:
\begin{itemize}
    \item Vision quests and shamanic practice
    \item Male puberty rites
    \item Medicinal treatment of injuries
\end{itemize}

Both groups understood the plant's dangerous properties and limited its use to ceremonial contexts administered by knowledgeable elders.

\subsubsection{Dream and Vision Seeking}

Across all groups, dreams and visions were considered sources of spiritual power (\textit{puha/pooha}). Various plants---tobacco, jimsonweed, and others---facilitated access to this power.

\subsection{Trade Networks}

The Saline Valley region functioned as a trade crossroads:
\begin{itemize}
    \item Shoshone traded salt from Saline Valley
    \item California Shoshone traded with Tubatulabal
    \item Tubatulabal connected to Yokuts of the Central Valley
    \item Eastern Mono obtained acorns through trade from Western Mono
\end{itemize}

%-----------------------------------------------------------------------
\section{Contemporary Significance}
%-----------------------------------------------------------------------

\subsection{Land Repatriation}

Recent land repatriation efforts have enabled tribal members to reconnect with ancestral plant practices:
\begin{itemize}
    \item 2023: Fay Creek Ranch returned to T\"ubatulabal Tribe
    \item Ongoing: Owens Valley Indian Water Commission works to restore indigenous irrigation and traditional food sources
\end{itemize}

\subsection{Cultural Revitalization}

Traditional plant knowledge persists through:
\begin{itemize}
    \item Annual pine nut ceremonies (Walker River, Schurz)
    \item Traditional basketry continuing among Numu artists
    \item Tribal language and story preservation connecting plants to cultural identity
\end{itemize}

\subsection{Historical Suppression and Revival}

In the early 20th century, institutions like the Stewart Indian School actively suppressed traditional ceremonies as ``heathenish'' and ``barbarous.'' Despite this suppression, ceremonies persisted underground, resurfacing through organizations like the Western Shoshone Sacred Lands Association (1972). Today, pinyon harvests remain sacred communal events reinforcing cultural resilience \citep{cambridge2024}.

%-----------------------------------------------------------------------
\section{Conclusion}
%-----------------------------------------------------------------------

The Indigenous peoples of the Saline Valley region developed sophisticated ethnobotanical knowledge systems adapted to the challenging environments of the Great Basin and Sierra Nevada. From the Kawaiisu's four sacred medicines to the Mono's advanced irrigation agriculture, from Tubatulabal jimsonweed ceremonies to Northern Paiute pine nut festivals, plants served not merely as resources but as spiritual beings deserving respect.

Common threads unite these diverse peoples: reliance on pi\~{n}on pine as a dietary staple, use of plants for vision-seeking and healing, sophisticated basketry traditions, and trade networks connecting mountain and desert communities. The Saline Valley's salt deposits placed it at the center of these exchange relationships.

Today, these traditions continue through ceremonial gatherings, land repatriation efforts, and cultural revitalization programs. As the Numu teach basket-makers to weave only good thoughts into their work, so too do these plant traditions weave Indigenous peoples to their ancestral landscapes.

%-----------------------------------------------------------------------
\bibliographystyle{apalike}
\bibliography{references}

\end{document}
