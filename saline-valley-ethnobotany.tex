\documentclass[12pt,letterpaper]{article}
\usepackage[utf8]{inputenc}
\usepackage[T1]{fontenc}
\usepackage{geometry}
\usepackage{natbib}
\usepackage{graphicx}
\usepackage{hyperref}
\usepackage{booktabs}
\usepackage{longtable}
\usepackage{array}
\usepackage{enumitem}

\geometry{margin=1in}

\title{Ethnobotany of the Saline Valley Region:\\
Plant Uses and Rituals Among Indigenous Peoples of the\\
Southern Great Basin and Eastern Sierra Nevada}

\author{Research Report for Ge136}

\date{\today}

\begin{document}

\maketitle

\begin{abstract}
This report documents traditional plant uses and rituals among five Indigenous groups with historical connections to the Saline Valley region of California: the Kawaiisu, Tubatulabal, Newe Sogobia (Western Shoshone), N\"u\"um\"u Wit\"u (Eastern Mono/Monache), and N\"u\"um\"u (Northern Paiute). Drawing on ethnographic literature and contemporary sources, this synthesis examines food plants, medicinal applications, ceremonial uses, and material culture across these interconnected Great Basin and Sierra Nevada peoples.
\end{abstract}

\tableofcontents
\newpage

%-----------------------------------------------------------------------
\section{Introduction}
%-----------------------------------------------------------------------

The Saline Valley, located in the northern Mojave Desert of eastern California, sits at a cultural crossroads between Great Basin and Sierra Nevada peoples. For millennia, Indigenous groups traversed this region, utilizing its diverse plant communities from hot desert scrublands to cold mountain forests. The Timbisha Shoshone maintained direct connections to Saline Valley, using its salt deposits as a trade commodity \citep{factcards2024}. Neighboring groups---the Kawaiisu to the south, Tubatulabal to the west, Eastern Mono to the northwest, and Northern Paiute to the north---shared linguistic ties (Numic language family) and ecological knowledge systems.

This report synthesizes ethnobotanical knowledge for five groups with cultural presence in or near the Saline Valley region:

\begin{enumerate}
    \item \textbf{Kawaiisu} --- Tehachapi Mountains and southern Sierra Nevada
    \item \textbf{Tubatulabal} --- Upper Kern River Valley
    \item \textbf{Newe Sogobia (Western Shoshone)} --- Great Basin, including Death Valley and Saline Valley
    \item \textbf{N\"u\"um\"u Wit\"u (Eastern Mono/Monache)} --- Owens Valley and eastern Sierra Nevada
    \item \textbf{N\"u\"um\"u (Northern Paiute)} --- Owens Valley to northern Nevada/Oregon
\end{enumerate}

All five groups developed sophisticated knowledge of native plants, understanding ``which plants and plant parts were useful for curing certain ailments, which produced colorful dyes, which would keep spirits away, and `which crops ripened when' in a particular locality'' \citep{rhode2002}.

%-----------------------------------------------------------------------
\section{The Kawaiisu}
%-----------------------------------------------------------------------

\subsection{Cultural Context}

The Kawaiisu traditionally occupied the Tehachapi Mountains and adjacent areas of the southern Sierra Nevada, with cultural connections extending into the western Mojave Desert. Maurice L. Zigmond's \textit{Kawaiisu Ethnobotany} (1981) remains the authoritative source on their plant knowledge \citep{zigmond1981}.

\subsection{Food Plants}

\subsubsection{Acorns and Pine Nuts}

The Kawaiisu gathered acorns from California scrub oak (\textit{Quercus berberidifolia}) and stored them in elevated granaries for long-term preservation. Pi\~{n}on pine nuts (\textit{Pinus monophylla}) served as another dietary staple \citep{ethnoherbalist2024}.

\subsubsection{Seeds and Greens}

\begin{itemize}
    \item \textbf{Chia} (\textit{Salvia columbariae}) --- Seeds parched and ground \citep{zigmond1981}
    \item \textbf{Dock} (\textit{Rumex} spp.) --- Seeds parched, pounded, and mixed into porridge \citep{zigmond1981}
    \item \textbf{Lomatium} (\textit{Lomatium utriculatum}) --- Roots consumed as food \citep{zigmond1981}
\end{itemize}

\subsection{Medicinal Plants}

\begin{longtable}{p{3.5cm}p{4cm}p{6cm}}
\toprule
\textbf{Plant} & \textbf{Scientific Name} & \textbf{Use} \\
\midrule
\endhead
Virgin River Brittlebush & \textit{Encelia virginensis} & Leaf and flower decoction as wash for rheumatic pain; also for horse injuries \\
Yerba Santa & \textit{Eriodictyon californicum} & Tea consumed instead of water for one month to treat gonorrhea \\
Lomatium & \textit{Lomatium utriculatum} & Root decoction as wash for broken limbs \\
California Juniper & \textit{Juniperus californica} & Berries eaten fresh or dried and pulverized \\
White Sage & \textit{Salvia apiana} & Multiple medicinal applications \\
\bottomrule
\caption{Kawaiisu Medicinal Plants}
\end{longtable}

\subsection{Material Culture}

\begin{itemize}
    \item \textbf{Cattail} (\textit{Typha} spp.) --- Leaves for thatching roofs and walls; flower down for bedding
    \item \textbf{Willow} (\textit{Salix} spp.) --- Poles for house and sweathouse construction; branches for bows
    \item \textbf{Buckbrush} (\textit{Ceanothus cuneatus}) --- Firewood; straight twigs for arrow foreshafts
    \item \textbf{California Juniper} --- Shredded bark for roof thatch; wood for kitchen tools and bows
\end{itemize}

\subsection{Ceremonial and Spiritual Uses}

\subsubsection{The Four Medicines}

According to Kawaiisu cosmology, in the Beginning, four medicines were given to the people: tobacco, nettles, red ants, and jimsonweed (\textit{Datura wrightii}). All four were used to induce dreams and visions as well as alleviate pain \citep{vredenburgh2024}.

\subsubsection{Jimsonweed (\textit{Datura wrightii})}

Jimsonweed held central importance in Kawaiisu spiritual practice. An infusion of the roots was drunk to obtain visions that could foretell the future, generally limited to winter when the root was considered less dangerous. The same infusion helped heal broken bones. For arthritis, roots were mashed and soaked, and sore limbs bathed in the preparation \citep{vredenburgh2024}.

Shamanic use of Datura was practiced by the Kawaiisu, and puberty rites for males involving the plant were also conducted \citep{netroots2024}.

\subsubsection{Tobacco}

Eating tobacco (\textit{Nicotiana bigelovii}) was another accepted procedure for gaining beneficial visions \citep{vredenburgh2024}.

%-----------------------------------------------------------------------
\section{The Tubatulabal}
%-----------------------------------------------------------------------

\subsection{Cultural Context}

The T\"ubatulabal have been stewards of the Kern Valley for thousands of years, with their traditional homeland encompassing the upper Kern River drainage in the southern Sierra Nevada. Erminie W. Voegelin's \textit{Tubatulabal Ethnography} (1938) provides the foundational ethnographic documentation \citep{voegelin1938}.

\subsection{Food Plants}

\subsubsection{Primary Staples}

\begin{itemize}
    \item \textbf{Acorns} --- Gathered from the Greenhorn Mountains in late fall, dried in the sun, stored in elevated granaries
    \item \textbf{Pi\~{n}on Nuts} --- Collected from eastern Sierra slopes in early fall, heated to open cones, dried, stored in stone-lined pits
\end{itemize}

\subsubsection{Other Plant Foods}

\begin{itemize}
    \item Seeds: chia, wild oats
    \item Leaves, bulbs, tubers, and roots
    \item Berries: juniper, manzanita, gooseberries, boxthorn
    \item Sugar crystals from honey dew cane stalks
\end{itemize}

Plants were prepared by boiling into mush, roasting, or baking in earth ovens. Berries could be pounded, mixed with water, shaped into cakes, dried, and stored for winter \citep{factcards2024tubatulabal}.

\subsection{Material Culture}

\subsubsection{Basketry}

Baskets were made from:
\begin{itemize}
    \item \textbf{Split willow branches} --- Primary structural material
    \item \textbf{Yucca roots} (\textit{Yucca}) --- Alternative structural material; red-colored pieces for design patterns
    \item \textbf{Devil's claw} (\textit{Proboscidea}) --- Black patterns on coiled baskets
\end{itemize}

Both twining and coiling methods were used, with designs applied only to coiled baskets. Baskets served for carrying food, sifting grains, cooking, and serving \citep{factcards2024tubatulabal}.

Estefana Miranda, a noted Tubatulabal elder, knew how to make ``flat round'' baskets used for both sifting and ceremonies, as well as processing pi\~{n}on nuts from Walker Pass, Kennedy Meadows, and Greenhorn Mountain areas.

\subsubsection{Musical Instruments}

\begin{itemize}
    \item Flutes made from elderberry (\textit{Sambucus}) stalks
    \item Various plant-fiber rattles
    \item Quill whistles
    \item Musical bows
\end{itemize}

\subsubsection{Hunting Equipment}

Many kinds of nets, traps, and snares were woven from plant fibers. Large nets were set up across canyons for communal rabbit drives \citep{factcards2024tubatulabal}.

\subsection{Ceremonial and Medicinal Uses}

\subsubsection{Jimsonweed}

Among the Tubatulabal, jimsonweed was believed to have once been a man who transformed himself into the plant to help the people cure sickness and find spiritual power. According to anthropologist Charles Smith, ``No form of worship was attached to the plant or its use. Rather, the plant's medicinal properties and its use in obtaining supernatural power and longevity were emphasized'' \citep{everyculture2024}.

Sacred Datura was applied topically or taken internally to treat:
\begin{itemize}
    \item Broken bones or sprains
    \item Associated pain and swelling
    \item Intestinal bloat and constipation
\end{itemize}

\subsubsection{Shamanism}

Both men and women could become shamans among the Tubatulabal. Male shamans had both curing and witching powers; female shamans had only witching power. The Tubatulabal believed in numerous spirits, both human and animal, all treated with respect \citep{factcards2024tubatulabal}.

\subsection{Contemporary Revival}

In 2023, Western Rivers Conservancy and the T\"ubatulabal Tribe preserved Fay Creek Ranch, a series of freshwater springs feeding the South Fork Kern River. This represents the first land repatriated to the tribe, allowing tribal members to reconnect with ancestral lands, hold ceremonies, and practice traditional plant gathering \citep{wrc2024}.

%-----------------------------------------------------------------------
\section{Newe Sogobia (Western Shoshone)}
%-----------------------------------------------------------------------

\subsection{Cultural Context}

The Western Shoshone, who call themselves Newe (``The People''), traditionally occupied the vast Great Basin territory they call Newe Sogobia. The Timbisha Shoshone (Death Valley/Panamint Shoshone) maintained direct connections to Saline Valley, Death Valley, and Panamint Valley \citep{factcards2024}.

\subsection{Food Plants}

\subsubsection{Pine Nuts}

Pine nuts took the place of acorns for the Shoshone, who had few oak trees in their territory. Nuts were gathered in early fall, with each family attempting to harvest enough to last through winter \citep{factcards2024}.

\subsubsection{Seeds and Mesquite}

Seeds from many grasses and plants were gathered in summer, roasted to dry, and ground into flour for storage. In the desert, mesquite (\textit{Prosopis}) pods were particularly important. The beans were pounded to make flour for flat cakes \citep{factcards2024}.

\subsubsection{Green Plants}

Fresh green plants were scarce in the desert. Those available (such as clover) required boiling and squeezing to remove bitter mineral salts from desert soil \citep{factcards2024}.

\subsection{Medicinal Plants}

Traditional Shoshone view their food as medicine, understanding that ``nature truly heals'' \citep{nativememory2024}.

\begin{longtable}{p{3.5cm}p{4cm}p{6cm}}
\toprule
\textbf{Plant} & \textbf{Scientific Name} & \textbf{Use} \\
\midrule
\endhead
Sagebrush & \textit{Artemisia tridentata} & Leaf tea to expel lung/stomach phlegm; induces sweating for colds \\
Black Hawthorn & \textit{Crataegus douglasii} & Berries, flowers, or leaves added to food for health \\
Canadian Goldenrod & \textit{Solidago canadensis} & ``Food as medicine'' --- consumed for health benefits \\
Silver Buffaloberry & \textit{Shepherdia argentea} & Cooked berries high in lycopene for cardiovascular health \\
Broadleaf Plantain & \textit{Plantago major} & Leaf poultice for wounds, especially slow-healing burns \\
\bottomrule
\caption{Western Shoshone Medicinal Plants}
\end{longtable}

The Shoshone and Gosiute prepared plant poultices to treat swelling, and root infusions for eye medicine \citep{redbuttegarden2024}.

\subsection{Trade Networks}

The California Shoshone maintained contacts with groups west of the Sierra Nevada, including trade relationships with the Tubatulabal and, through them, the Yokuts of the Central Valley. Salt from Saline Valley served as a valuable trade commodity \citep{factcards2024}.

\subsection{Ceremonial Practices}

\subsubsection{Pine Nut Ceremony}

For the Shoshone, gathering pine nuts is always preceded by ceremony expressing gratitude, performed with drum, song, and dance. Women dance displaying harvest tools---roasting and winnowing baskets whose designs have remained unchanged since antiquity. A traditional blessing is given before harvest begins \citep{nativememory2024harvest}.

The Yomba Shoshone Tribe preserves sacred ceremonies rooted in Great Basin traditions of harmony with land, spirits, and ancestors. These rituals blend vision quests, communal dances, and seasonal observances, emphasizing spiritual power gained through dreams and nature reverence \citep{yomba2024}.

\subsubsection{Musical Instruments}

Some Shoshone people used a four-holed flute made from elderberry wood for festival music \citep{factcards2024}.

%-----------------------------------------------------------------------
\section{N\"u\"um\"u Wit\"u (Eastern Mono/Monache)}
%-----------------------------------------------------------------------

\subsection{Cultural Context}

The Mono (N\"u\"um\"u, meaning ``People'') traditionally inhabit the central Sierra Nevada, Eastern Sierra south of Bridgeport, Mono Basin, and adjacent Great Basin areas. Two main dialect groups exist: Eastern Mono (Owens Valley Paiute) in the Owens River Valley, and Western Mono (Monachi) on the western Sierra slopes \citep{wikipedia2024mono}.

\subsection{Food Plants}

\subsubsection{Eastern Mono (Owens Valley Paiute)}

The Eastern Mono emphasized:
\begin{itemize}
    \item \textbf{Pine nuts} from pi\~{n}on groves --- roasted, shelled, stored or ground into flour
    \item \textbf{Seeds} from grasses and sages --- beaten into baskets
    \item Acorns obtained primarily through trade (scarce in drier eastern territories)
\end{itemize}

\subsubsection{Western Mono (Monache)}

The Western Mono, occupying Sierra Nevada foothills at 3,000--7,000 feet elevation, relied heavily on:
\begin{itemize}
    \item \textbf{Acorns} --- primary dietary staple, processed by leaching tannins through water immersion, ground into meal for mush or cakes
\end{itemize}

The Western Mono inhabited oak woodlands enabling specialized acorn-based adaptations \citep{wikipedia2024mono}.

\subsection{Indigenous Irrigation Agriculture}

The N\"u\"um\"u of Owens Valley developed North America's most sophisticated indigenous irrigation system. For millennia, they spread water across the land to sustain plants and animals, calling their homeland Payah\"u\"unad\"u (``the land of flowing water'') \citep{oviwc2024}.

\subsubsection{Water Management System}

\begin{itemize}
    \item Each year, the people elected a \textbf{tuvaij\"u} (head irrigator) to assess lands and decide where to spread water
    \item Ditches and waterways were dug by hand using the \textbf{pavodo}---an eight-foot hardwood pole
    \item Irrigated areas were alternated to allow land recovery
    \item Over sixty miles of indigenous waterworks irrigated the valley
\end{itemize}

This vegeculture system included irrigation of a variety of seed-bearing plants, representing ``perhaps the best instance in North America of a group that developed its own system of vegeculture'' \citep{steward1930}.

\subsection{Basketry}

Mono women's basketry ranks among the finest in the world. Their extensive plant knowledge remains relevant today because ``native plants are better suited to our environment'' \citep{altapeak2024}.

\subsection{Ceremonial Life}

The Owens Valley Paiute participated in round dances and held annual harvest festivals. Elaborate puberty ceremonies were conducted for girls. Mourning was expressed through ``The Cry,'' a ceremony Yuman in origin that included ritual face washing after a year of mourning \citep{britannica2024}.

%-----------------------------------------------------------------------
\section{N\"u\"um\"u (Northern Paiute)}
%-----------------------------------------------------------------------

\subsection{Cultural Context}

The Northern Paiute (Numu) aboriginal lands span parts of California, Nevada, eastern Oregon, and southern Idaho. The landscape includes flat bottomland, painted desert, canyon, plateau, and mountain areas---all holding enormous spiritual significance \citep{nativeland2024paiute}.

\subsection{Spiritual Worldview: Puha/Pooha}

Northern Paiute believe that power (\textit{pooha} or \textit{puha}) resides in natural objects including animals, plants, and geographical features. Puha was understood as a pervasive supernatural power in the sun, moon, stars, thunder, wind, water, rocks, and plants. It could be acquired through dreams, visions, or rituals, enabling individuals---particularly shamans---to manipulate it for healing or finding lost objects \citep{nevadasindian2024}.

\subsection{Food Plants}

\subsubsection{Seeds and Roots}

Important plant foods included:
\begin{itemize}
    \item \textbf{Seeds}: Indian ricegrass, sunflowers
    \item \textbf{Roots and bulbs}: sego lily, bitterroot, yampa, biscuitroot
    \item \textbf{Seasonal berries}: chokecherries, currants
\end{itemize}

Root harvesting occurred during April--May ``root camps'' lasting approximately six weeks. Over 30 species are documented in ethnobotanical records, with emphasis on knowledge of habitat timing to maximize yields without depleting stands \citep{britannica2024paiute}.

\subsubsection{Preservation}

Air-drying was the primary preservation technique, enabling storage for winter consumption.

\subsection{Material Culture}

\subsubsection{Tule and Cattails}

Numu people traditionally used tule (\textit{Schoenoplectus acutus}) to make:
\begin{itemize}
    \item Gathering bags
    \item Matting
    \item Hunting decoys
    \item Boats (secured with cattails)
\end{itemize}

\subsubsection{Basketry}

A few Numu artists continue traditional basket-making, which requires ``an intimate connection with Numu plants, land, and language.'' Numu stories teach basket-makers to weave only good thoughts into baskets as they create them \citep{nevadasindian2024}.

\subsection{Pine Nut Festival}

Every third weekend of September, several hundred American Indians and visitors gather at the Walker River Paiute Tribe Reservation in Schurz for the Pine Nut Festival. Events include:
\begin{itemize}
    \item Powwow
    \item Hand games
    \item Pine nut blessing and dance
\end{itemize}

During the pine nut dance, dancers move on sacred ground around a pine tree with pi\~{n}on offerings for the blessing. This spiritual ceremony dates back over a hundred years to when pine nuts were winter subsistence for Great Basin Indians. Today, the dance honors ancestors, preserves native traditions, and revives spiritual practices \citep{nativememory2024harvest}.

%-----------------------------------------------------------------------
\section{Comparative Analysis: Shared Plant Knowledge}
%-----------------------------------------------------------------------

\subsection{Common Food Plants Across Groups}

\begin{longtable}{p{3cm}p{2cm}p{2cm}p{2cm}p{2cm}p{2cm}}
\toprule
\textbf{Plant} & \textbf{Kawaiisu} & \textbf{Tubatulabal} & \textbf{W. Shoshone} & \textbf{E. Mono} & \textbf{N. Paiute} \\
\midrule
\endhead
Pi\~{n}on Pine & $\checkmark$ & $\checkmark$ & $\checkmark$ & $\checkmark$ & $\checkmark$ \\
Acorns & $\checkmark$ & $\checkmark$ & Trade & $\checkmark$* & Limited \\
Chia & $\checkmark$ & $\checkmark$ & --- & --- & --- \\
Juniper & $\checkmark$ & $\checkmark$ & $\checkmark$ & --- & --- \\
Mesquite & --- & --- & $\checkmark$ & --- & --- \\
Wild seeds & $\checkmark$ & $\checkmark$ & $\checkmark$ & $\checkmark$ & $\checkmark$ \\
\bottomrule
\caption{Comparative Food Plant Use (*Western Mono primary; Eastern Mono through trade)}
\end{longtable}

\subsection{Shared Ceremonial Elements}

\subsubsection{Pine Nut Ceremonies}

All five groups participated in some form of pine nut harvest ceremony, though intensity varied with ecological availability. These ceremonies persist today as affirmations of cultural heritage.

\subsubsection{Jimsonweed Use}

The Kawaiisu and Tubatulabal both used jimsonweed (\textit{Datura wrightii}) for:
\begin{itemize}
    \item Vision quests and shamanic practice
    \item Male puberty rites
    \item Medicinal treatment of injuries
\end{itemize}

Both groups understood the plant's dangerous properties and limited its use to ceremonial contexts administered by knowledgeable elders.

\subsubsection{Dream and Vision Seeking}

Across all groups, dreams and visions were considered sources of spiritual power (\textit{puha/pooha}). Various plants---tobacco, jimsonweed, and others---facilitated access to this power.

\subsection{Trade Networks}

The Saline Valley region functioned as a trade crossroads:
\begin{itemize}
    \item Shoshone traded salt from Saline Valley
    \item California Shoshone traded with Tubatulabal
    \item Tubatulabal connected to Yokuts of the Central Valley
    \item Eastern Mono obtained acorns through trade from Western Mono
\end{itemize}

%-----------------------------------------------------------------------
\section{Contemporary Significance}
%-----------------------------------------------------------------------

\subsection{Land Repatriation}

Recent land repatriation efforts have enabled tribal members to reconnect with ancestral plant practices:
\begin{itemize}
    \item 2023: Fay Creek Ranch returned to T\"ubatulabal Tribe
    \item Ongoing: Owens Valley Indian Water Commission works to restore indigenous irrigation and traditional food sources
\end{itemize}

\subsection{Cultural Revitalization}

Traditional plant knowledge persists through:
\begin{itemize}
    \item Annual pine nut ceremonies (Walker River, Schurz)
    \item Traditional basketry continuing among Numu artists
    \item Tribal language and story preservation connecting plants to cultural identity
\end{itemize}

\subsection{Historical Suppression and Revival}

In the early 20th century, institutions like the Stewart Indian School actively suppressed traditional ceremonies as ``heathenish'' and ``barbarous.'' Despite this suppression, ceremonies persisted underground, resurfacing through organizations like the Western Shoshone Sacred Lands Association (1972). Today, pinyon harvests remain sacred communal events reinforcing cultural resilience \citep{cambridge2024}.

%-----------------------------------------------------------------------
\section{Conclusion}
%-----------------------------------------------------------------------

The Indigenous peoples of the Saline Valley region developed sophisticated ethnobotanical knowledge systems adapted to the challenging environments of the Great Basin and Sierra Nevada. From the Kawaiisu's four sacred medicines to the Mono's advanced irrigation agriculture, from Tubatulabal jimsonweed ceremonies to Northern Paiute pine nut festivals, plants served not merely as resources but as spiritual beings deserving respect.

Common threads unite these diverse peoples: reliance on pi\~{n}on pine as a dietary staple, use of plants for vision-seeking and healing, sophisticated basketry traditions, and trade networks connecting mountain and desert communities. The Saline Valley's salt deposits placed it at the center of these exchange relationships.

Today, these traditions continue through ceremonial gatherings, land repatriation efforts, and cultural revitalization programs. As the Numu teach basket-makers to weave only good thoughts into their work, so too do these plant traditions weave Indigenous peoples to their ancestral landscapes.

%-----------------------------------------------------------------------
\bibliographystyle{apalike}
\bibliography{references}

\end{document}
