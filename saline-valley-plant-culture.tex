\documentclass[11pt,letterpaper]{article}
\usepackage[utf8]{inputenc}
\usepackage[T1]{fontenc}
\usepackage{geometry}
\usepackage{natbib}
\usepackage{graphicx}
\usepackage{booktabs}
\usepackage{longtable}
\usepackage{array}
\usepackage{float}
\usepackage{wrapfig}
\usepackage{caption}
\usepackage{hyperref}
\usepackage{xcolor}

\geometry{margin=1in}
\graphicspath{{figures/extracted/}{figures/maps/}}

\title{Desert Plant Healing, Rituals, and Material Culture\\
\large A Field Guide to Indigenous Ethnobotany of the Saline Valley Region}

\author{Isaac N. Aguilar\\
\small Ge136 Field Studies}

\date{\today}

\begin{document}

\maketitle

%-----------------------------------------------------------------------
\section{Introduction: Land as Relation}
%-----------------------------------------------------------------------

The Southern Deserts of California are often mischaracterized as overlooked wilderness. Yet California Indians did not distinguish between managed land and wild ``preserves.'' A hands-off management paradigm perpetuated by colonial hegemony has promoted feral landscapes that have not been taken care of by humans for a long time and are becoming inhospitable to life. When intimate interaction with the plants and animals ceases, the continuity of generational knowledge is broken and the land becomes ``wilderness.''

This document synthesizes ethnobotanical knowledge for the Saline Valley region, where multiple Indigenous groups developed sophisticated relationships with the desert flora over millennia (Table \ref{tab:groups}). The Timbisha Shoshone---whose name derives from \textit{tümpisa}, the red ochre paint found in Death Valley---maintained primary residence in the Saline Valley, Death Valley, and Panamint Valley corridor. Neighboring peoples shared linguistic roots in the Numic branch of the Uto-Aztecan language family, facilitating trade, intermarriage, and the transmission of ecological knowledge across the region. Their knowledge systems integrated food procurement, healing practices, ceremonial life, and material production into a unified understanding of the landscape.

\begin{table}[H]
\centering
\caption{Indigenous Peoples of the Saline Valley Region}
\label{tab:groups}
\small
\begin{tabular}{@{}p{3.2cm}p{2.8cm}p{3cm}p{5cm}@{}}
\toprule
\textbf{Group} & \textbf{Autonym} & \textbf{Language} & \textbf{Traditional Territory} \\
\midrule
Timbisha Shoshone & T\"umpisa & Central Numic (Shoshoni) & Saline Valley, Death Valley, Panamint Valley, Grapevine \& Cottonwood Mtns \\[0.5em]
Owens Valley Paiute & N\"u\"um\"u & Western Numic (Mono) & Owens Valley, eastern Sierra Nevada escarpment, Fish Lake Valley \\[0.5em]
Western Shoshone & Newe & Central Numic (Shoshoni) & Central Great Basin, Deep Springs Valley, Eureka Valley \\[0.5em]
Northern Paiute & Numu (Paviotso) & Western Numic (N. Paiute) & Mono Lake basin, northern Owens Valley, western Nevada \\[0.5em]
Kawaiisu & Nuooah & Southern Numic & Tehachapi Mtns, Piute Mtns, southern Sierra Nevada foothills \\[0.5em]
Tubatulabal & P\=ah\=ahn-\=ap\v{i}l & Tubatulabalic (isolate) & Upper Kern River Valley, Kern River drainage \\[0.5em]
Koso/Panamint & Koso & Central Numic (Shoshoni) & Coso Range, Little Lake, Argus Range \\
\bottomrule
\end{tabular}
\end{table}

All these groups---except the Tubatulabal, who represent an earlier Uto-Aztecan stratum---speak Numic languages that spread across the Great Basin approximately 1,000 years ago. This ``Numic expansion'' carried shared mythological traditions, subsistence strategies, and plant knowledge that created cultural continuity across vast distances. Saline Valley, with its salt deposits and position between the Sierra Nevada and Death Valley, served as a nexus where these peoples converged for trade, ceremony, and seasonal resource exploitation.

\begin{figure}[H]
\centering
\includegraphics[width=0.9\textwidth]{saline_valley_territories_geo.png}
\caption{Indigenous territories of the Saline Valley region overlaid on Basin and Range topography. The Timbisha Shoshone occupied the central corridor including Saline Valley, Death Valley, and surrounding ranges. The characteristic north-south trending mountain ranges created distinct ecological zones exploited through seasonal migration.}
\label{fig:territories}
\end{figure}

%-----------------------------------------------------------------------
\section{Cultural Context}
%-----------------------------------------------------------------------

The Indigenous peoples of the Saline Valley region developed sophisticated social organizations, maintained extensive trade networks, and elaborated rich cosmovisions that gave meaning to their relationship with the natural world. Social organization followed a gradient of resource availability: ``family'' groups in resource-poor areas and ``band'' groups where abundance permitted larger aggregations. The Timbisha Shoshone exemplify this pattern---Julian Steward documented populations of approximately 65 persons in Saline Valley, 100 in northern Panamint Valley, and 150--160 in the Little Lake and Coso Range areas. Leadership remained fundamentally egalitarian; shamans (\textit{puagants} or \textit{poha'ganti}) wielded influence through their ability to manipulate \textit{puha} (spiritual power), but coercive political authority was absent.

Seasonal rounds structured the year. Both oral tradition and archaeological evidence confirm that Timbisha people moved between semi-permanent villages throughout greater Death Valley. During summer months, when valley floor temperatures became unbearable, family groups migrated to cooler mountain elevations. Seasonal camps at Wildrose Canyon, Hunter Mountain, and the Grapevine Mountains served as communal gathering sites where fifty to one hundred individuals harvested pi\~{n}on nuts, seeds, roots, and berries. Hunting focused on bighorn sheep (\textit{tuhu}), mule deer, and jackrabbit. These summer communes served critical social functions: relatives reacquainted themselves, marriages were arranged, and ceremonial knowledge was transmitted across generations.

With the onset of cool weather, groups descended to valley floor villages where mesquite groves and springs provided winter resources. Villages at Furnace Creek (\textit{tümpisa}), Saline Valley, and Grapevine Canyon supported year-round occupation. The salt deposits of Saline Valley were collected by the Timbisha and traded with the Owens Valley Paiute for brown-ware pottery, obsidian, and other goods. Shell beads from the Pacific Coast, traded through Yokuts and Kawaiisu intermediaries, have been found in Timbisha sites---evidence of exchange networks spanning hundreds of miles.

%-----------------------------------------------------------------------
\section{Plant Foods}
%-----------------------------------------------------------------------

\subsection{Tree Crops: Pi\~{n}on, Oak, and Mesquite}

The annual pi\~{n}on pine (\textit{Pinus monophylla}) harvest concentrated dispersed family groups onto ceremonial harvesting grounds, transforming subsistence labor into communal celebration. Pi\~{n}on is notable for its high carbohydrate content (54\%), unsaturated fatty acids (oleic, linoleic, linolenic), and presence of all twenty amino acids \citep{lanner1981}. The green cones are roasted to release seeds, remove resin, and enhance digestibility. Seeds are winnowed in basket trays bearing elegant, generational designs---the rhythmic tossing motion separating nutmeat from chaff while scattering some seeds to ensure future harvests.

Acorns---though oaks do not grow in Saline Valley proper---were obtained through trade with western Sierra groups or gathered during seasonal movements to higher elevations. Canyon live oak (\textit{Quercus chrysolepis}) occurs on the western slopes of the Inyo Mountains, and scrub oak (\textit{Q. berberidifolia}) in the Tehachapi and southern Sierra foothills. The processing of acorns required elaborate technology to render them edible. Bitter polyphenolic tannins interfere with nutrient absorption and protein digestion; removal involves drying acorns for up to a year, pounding the dried nuts into meal using bedrock mortars, and leaching the flour with repeated water applications. Acorns with higher tannin content required leaching for up to two weeks. Processing temperatures below 150\textdegree F preserve a starch analogous to gluten, enabling the meal to be formed into cakes for baking \citep{bainbridge1986}. Both acorns and pi\~{n}on nuts were stored in elevated granaries lined with aromatic plants such as big sagebrush (\textit{Artemisia tridentata}) or California mugwort (\textit{A. douglasiana}) to repel insects.

Mesquite (\textit{Prosopis glandulosa}) dominated the valley floor villages that sustained winter populations. The pods contain 25--30\% sugar and significant protein, ground into a sweet flour called \textit{p\'{e}chita} that could be stored indefinitely. Mesquite flour was mixed with water to form cakes or beverages, providing reliable calories when other resources were scarce. The wood fueled cooking fires and the thorny branches formed corrals and windbreaks.

\subsection{Seeds: The Staff of Desert Life}

Seeds formed the dietary foundation across the region, their diversity providing security against the failure of any single crop. Chia (\textit{Salvia columbariae}) seeds were parched and ground into a mucilaginous flour rich in omega-3 fatty acids. Willow dock (\textit{Rumex salicifolius}) seeds were ground into porridge. Indian ricegrass (\textit{Eriocoma hymenoides})---one of the most important seed foods of the Great Basin---thrived on the sandy soils of Saline Valley; its seeds were parched, winnowed, and ground into flour or added whole to stews. Multiple saltbush species provided both edible seeds and leaves: shadscale (\textit{Atriplex confertifolia}) on the alkaline valley floors, cattle saltbush (\textit{A. polycarpa}) on alluvial fans, and desert holly (\textit{A. hymenelytra}) whose young leaves were eaten as greens and whose silvery foliage was used ceremonially.

To prepare seeds for consumption, they were lightly roasted in a tightly woven winnowing basket with live coals. The heat enhanced flavor while loosening husks and chaff, which separated from the heavier seeds as the weaver tossed them skillfully in the air. Seeds could then be eaten whole, pounded into flour with a milling stone, or added to stews and gruels. Some seeds inevitably flew too far or bounced from gathering baskets, but this loss was not lamented---some seeds needed to fall on the ground to produce the next year's crop. This incidental scattering might not sustain plant populations over time; for this reason, Indigenous peoples deliberately saved seeds and broadcast them in appropriate habitats, a practice that blurs Western distinctions between gathering and agriculture.

\subsection{Roots, Shoots, and Greens}

The starchy roots of Mojave desertparsley (\textit{Lomatium mohavense}) were dug with hardwood digging sticks and eaten raw, roasted, or dried for storage. Southern cattail (\textit{Typha domingensis}) provided multiple resources from a single plant: the starchy rhizomes were roasted or dried and ground into flour; the protein-rich pollen was collected in spring and used as flour; the leaves were woven into mats, shelters, and the tule boats that navigated Owens Lake. Young shoots of cattail were eaten raw as a vegetable.

Joshua tree (\textit{Yucca brevifolia}) flower stalks were roasted in earth ovens, yielding a sweet, fibrous food. The fruit was also eaten, the seeds ground into meal, and the plant's utility extended far beyond subsistence into material culture. The tough leaf fibers provided cordage, and the roots produced a soap rich in saponins.

%-----------------------------------------------------------------------
\section{Healing Plants}
%-----------------------------------------------------------------------

The same observational acuity that guided food procurement extended to medicinal plants. Healing knowledge was embedded in cosmological understanding---illness often resulted from spiritual imbalance, and plant medicines addressed both physical symptoms and their underlying causes.

\subsection{Big Sagebrush (\textit{Artemisia tridentata})}

The dominant aromatic shrub of the Great Basin, big sagebrush pervaded Indigenous healing practice. The volatile oils---camphor, thujone, 1,8-cineole---produce the plant's distinctive fragrance and its therapeutic effects. Leaves were burned and the smoke inhaled for respiratory congestion, sinus infections, and colds; the warming sensation results from activation of TRPV3 receptors in nasal passages. Leaf poultices treated wounds, the antimicrobial terpenes preventing infection. A tea from the leaves addressed stomach complaints and was used as a general tonic. Modern analysis confirms anti-inflammatory activity through inhibition of iNOS and NF-$\kappa$B pathways \citep{adams2012}.

\subsection{Yerba Santa (\textit{Eriodictyon} spp.)}

``Holy herb''---the Spanish name reflects the reverence this plant commanded. Though yerba santa grows primarily on the western slopes of the Sierra Nevada rather than in Saline Valley itself, its medicinal importance ensured wide distribution through trade networks. The resinous leaves were chewed, smoked, or brewed into tea for asthma, bronchitis, and tuberculosis. The flavonoid eriodictyol and related compounds exhibit expectorant properties, thinning mucus and facilitating its clearance from airways. Yerba santa was also applied as a poultice to wounds and used in sweat lodge ceremonies for purification. The Kawaiisu and Tubatulabal, with direct access to Sierra habitats, likely served as primary suppliers to desert groups.

\subsection{Willow (\textit{Salix} spp.)}

Willows demanded extensive environmental knowledge for their use. Branches are traditionally gathered in fall and dried through the cold winter air. The bark contains salicin, a white crystalline glucoside that the body converts to salicylic acid---the original ingredient in aspirin \citep{bocek1984}. Willow bark tea treated headaches, fever, and inflammatory conditions. The flexible young shoots served multiple purposes: willow withes bound the tule bundles of watercraft \citep{bryant1985}, while shoots provided material for toys, musical instruments, fish traps, and the foundational warps of coiled baskets.

\subsection{Mormon Tea (\textit{Ephedra nevadensis})}

The jointed green stems of Mormon tea were brewed into a stimulating beverage throughout the Great Basin. Unlike Asian \textit{Ephedra} species, the Nevada species contains negligible ephedrine, but the tea was valued for treating kidney ailments, venereal disease, and as a general tonic. The plant was widely traded and figures prominently in archaeological sites.

\subsection{Creosote Bush (\textit{Larrea tridentata})}

Creosote dominates the lower Mojave elevations, its resinous leaves releasing the distinctive scent of desert rain. A tea from the leaves treated respiratory infections, arthritis, rheumatism, and gastrointestinal complaints. Modern analysis reveals nordihydroguaiaretic acid (NDGA), a potent antioxidant that inhibits lipoxygenase enzymes and activates NRF2 cytoprotective pathways \citep{rahman2011}. The Cahuilla and Panamint also collected the amber-colored gum deposited by lac insects on creosote bark; when mixed withite stone and heated, this sticky resin fastened stone arrowheads to their shafts and mended cracked pottery \citep{purdy1976}.

\subsection{Juniper (\textit{Juniperus osteosperma})}

Utah juniper, often found growing with pi\~{n}on pine in the woodland belt, provided medicine, food, and ceremonial materials. The berries were eaten raw or dried; a tea from the berries and leaves treated colds, coughs, and kidney problems. Juniper branches were burned as incense for purification ceremonies, the aromatic smoke cleansing spaces and individuals. The shredded bark served as tinder, insulation, and padding for cradleboards.

%-----------------------------------------------------------------------
\section{Sacred Plants and Ritual}
%-----------------------------------------------------------------------

\subsection{Desert Tobacco (\textit{Nicotiana obtusifolia})}

Desert tobacco, native to the Saline Valley region, was cultivated---not merely gathered---by tribes throughout the area. The earth was loosened with digging sticks, and leaves were pruned to enhance their size. North-facing slopes were selected, and Thompson (1916) recorded the method: practitioners would ``pile brush over the ground and then burn it, which would leave the ground with a loose layer of wood ashes. Over this they would sow the seed and protect the crop by putting around it a brush fence. From year to year they would select from the best stalks, seed for the next year, and kept for a long time.'' Tobacco was smoked in ceremonies, offered to spirits, and applied medicinally to cuts. As an emetic, it induced the vomiting that purified participants before ritual undertakings.

\subsection{Datura (\textit{Datura wrightii})}

Sacred datura, also called jimsonweed or toloache, held profound ceremonial significance throughout Southern California and the Great Basin. The plant contains tropane alkaloids---scopolamine, hyoscyamine, atropine---that induce vivid hallucinations, altered perception, and communion with the spirit world. Datura was central to male puberty rites, vision quests, and shamanic practice.

Archaeological confirmation of Datura use comes from Pinwheel Cave in the western Transverse Ranges, where Robinson et al. (2020) identified \textit{Datura} quids (chewed plant material) in association with distinctive rock art. The ``pinwheel'' motif painted on the cave ceiling may represent the visual distortions---spiraling patterns, radial forms---experienced during Datura intoxication \citep{robinson2020}. The deliberate placement of quids in rock crevices suggests ritual deposition following ingestion.

Datura use required expert guidance. The alkaloid content varies dramatically between plants and plant parts; dosing errors could prove fatal. Knowledge of preparation, dosage, and ceremonial context was transmitted carefully across generations, embedded within the broader cosmological framework that gave the experience meaning.

\subsection{Sagebrush as Sacred Plant}

While white sage (\textit{Salvia apiana}) grows in southern California and was obtained through trade for ceremonial use, the locally abundant big sagebrush (\textit{Artemisia tridentata}) served parallel purification functions throughout the Great Basin. Sagebrush was burned for smudging---the aromatic smoke purifying spaces, objects, and individuals before ceremonies. The pungent volatile oils (camphor, thujone) created a sensory environment distinct from ordinary experience, marking ritual time and space. Sagebrush bark was also used to wrap sacred objects and line ceremonial spaces.

%-----------------------------------------------------------------------
\section{Material Culture}
%-----------------------------------------------------------------------

\subsection{Basketry}

A well-equipped kitchen contained basketry pots, pans, and serving dishes: some winnowed and parched seeds, others held stone-boiled soup and strained manzanita cider. Baskets held symbolic meaning in native cultures and were often gifted to strengthen kinship ties. Their production demanded intimate knowledge of plant ecology and phenology.

Native peoples took advantage of basketry plants' natural ability to sprout vigorously after disturbance. Burning or mechanical pruning stimulated production of the long, straight shoots required for warp and weft (Figure \ref{fig:basketry}). To rely on natural lightning fires for this purpose would be uncertain; California Indians actively managed vegetation to obtain sufficient regular supplies of materials. The approximately 25 weavers in a prehistoric village of 100 persons might harvest on the order of 250,000 shoots in a single season \citep{bates1990}.

Preferred basketry materials varied by tradition and application. Arroyo willow (\textit{Salix lasiolepis}) provided flexible warp rods and was the primary basketry willow of the region. Deergrass (\textit{Muhlenbergia rigens}) offered foundation bundles for coiled baskets. Design materials were often obtained through trade: redbud (\textit{Cercis occidentalis}) bark from the western Sierra contributed red-brown design elements, while Joshua tree (\textit{Yucca brevifolia}) root fibers and the split leaves of rushes provided local alternatives. The extensive trade networks documented archaeologically ensured access to materials beyond local habitats.

\begin{figure}[H]
\centering
\includegraphics[width=0.6\textwidth]{grinding-1.png}
\caption{Bedrock mortars at Indian Grinding Rock State Historic Park preserve evidence of intensive acorn and seed processing. Such sites were returned to across generations.}
\label{fig:basketry}
\end{figure}

\subsection{Fiber and Cordage}

Joshua tree (\textit{Yucca brevifolia}) provided the strongest plant fibers available in the region. Leaves were soaked, pounded, and the fibers extracted for cordage, nets, sandals, and snares. The roots produced a soap rich in saponins, used for washing hair and hides. Milkweed (\textit{Asclepias} spp.) and Indian hemp (\textit{Apocynum cannabinum}) yielded finer fibers twisted into string for nets and bags.

\subsection{Adhesives and Implements}

Pine pitch from pi\~{n}on and other conifers was heated and applied to seal baskets for water transport, haft tools, and waterproof seams. The creosote lac described above served similar purposes. Wooden implements---digging sticks, fire drills, bows, arrow shafts---were crafted from specific woods selected for their properties: mountain mahogany (\textit{Cercocarpus}) for its density and hardness, willow for its flexibility, juniper for its rot resistance.

%-----------------------------------------------------------------------
\section{Conclusion: Integration and Continuity}
%-----------------------------------------------------------------------

The ethnobotanical knowledge documented here represents fragments of integrated systems that unified subsistence, medicine, ceremony, and craft into coherent relationships with place. Each plant existed within webs of ecological, social, and spiritual meaning. Pi\~{n}on was not merely food but the occasion for community gathering and the transmission of cultural knowledge. Datura was not merely a hallucinogen but a vehicle for spiritual transformation guided by generations of accumulated wisdom. Willow was not merely medicine but material for the baskets that processed food, the boats that crossed lakes, and the fish traps that supplemented diet.

This integration challenges Western categories that separate nature from culture, sustenance from spirit, utility from meaning. The ``wilderness'' of Saline Valley was never wild to its Indigenous inhabitants---it was home, garden, pharmacy, and temple, maintained through continuous relationship across countless generations.

%-----------------------------------------------------------------------
\bibliographystyle{apalike}
\begin{thebibliography}{99}

\bibitem[Adams, 2012]{adams2012}
Adams, R.P. (2012).
\newblock Identification of essential oil components by gas chromatography/mass spectrometry.
\newblock \textit{Journal of Essential Oil Research}, 24(4):341--358.

\bibitem[Bainbridge, 1986]{bainbridge1986}
Bainbridge, D.A. (1986).
\newblock Use of acorns for food in California: Past, present, future.
\newblock \textit{Proceedings of the Symposium on Multiple-Use Management of California's Hardwood Resources}, pp. 453--458.

\bibitem[Bates \& Lee, 1990]{bates1990}
Bates, C.D. and Lee, M.J. (1990).
\newblock Tradition and innovation: A basket history of the Indians of the Yosemite-Mono Lake area.
\newblock Yosemite Association, Yosemite National Park.

\bibitem[Bocek, 1984]{bocek1984}
Bocek, B.R. (1984).
\newblock Ethnobotany of Costanoan Indians, California, based on collections by John P. Harrington.
\newblock \textit{Economic Botany}, 38(2):240--255.

\bibitem[Bryant, 1985]{bryant1985}
Bryant, E.H. (1985).
\newblock What I saw in California.
\newblock University of Nebraska Press (reprint).

\bibitem[Kroeber, 1925]{kroeber1925}
Kroeber, A.L. (1925).
\newblock Handbook of the Indians of California.
\newblock Bureau of American Ethnology Bulletin 78, Government Printing Office, Washington, D.C.

\bibitem[Lanner, 1981]{lanner1981}
Lanner, R.M. (1981).
\newblock The Pi\~{n}on Pine: A Natural and Cultural History.
\newblock University of Nevada Press, Reno.

\bibitem[Purdy, 1976]{purdy1976}
Purdy, B.A. (1976).
\newblock American Indian life in prehistoric Florida.
\newblock \textit{The Florida Anthropologist}, 29:75--87.

\bibitem[Rahman et al., 2011]{rahman2011}
Rahman, S., Ansari, R.A., Rehman, H., Parvez, S., and Raisuddin, S. (2011).
\newblock Nordihydroguaiaretic acid from creosote bush (\textit{Larrea tridentata}) mitigates 12-O-tetradecanoylphorbol-13-acetate-induced inflammatory and oxidative stress responses.
\newblock \textit{Evidence-Based Complementary and Alternative Medicine}, 2011:nep076.

\bibitem[Robinson et al., 2020]{robinson2020}
Robinson, D.W., Brown, K., McMenemy, M., et al. (2020).
\newblock Datura quids at Pinwheel Cave, California, provide unambiguous confirmation of the ingestion of hallucinogens at a rock art site.
\newblock \textit{Proceedings of the National Academy of Sciences}, 117(49):31207--31212.

\bibitem[Steward, 1933]{steward1933}
Steward, J.H. (1933).
\newblock Ethnography of the Owens Valley Paiute.
\newblock \textit{University of California Publications in American Archaeology and Ethnology}, 33(3):233--350.

\bibitem[Thompson, 1916]{thompson1916}
Thompson, L. (1916).
\newblock Notes on California Indians.
\newblock \textit{American Anthropologist}, 18(3):371--380.

\bibitem[Zigmond, 1981]{zigmond1981}
Zigmond, M.L. (1981).
\newblock Kawaiisu Ethnobotany.
\newblock University of Utah Press, Salt Lake City.

\end{thebibliography}

%-----------------------------------------------------------------------
\appendix
\section{Species Occurrence Verification}
%-----------------------------------------------------------------------

The following table summarizes iNaturalist observations (accessed February 2026) within 30 km of Saline Valley (36.75\textdegree N, 117.85\textdegree W), verifying local occurrence of species discussed in this report. Species obtained primarily through trade are noted.

\begin{small}
\begin{longtable}{@{}p{4.5cm}p{3.5cm}cp{4cm}@{}}
\toprule
\textbf{Species} & \textbf{Common Name} & \textbf{Obs.} & \textbf{Notes} \\
\midrule
\endfirsthead
\multicolumn{4}{c}{\textit{Continued from previous page}} \\
\toprule
\textbf{Species} & \textbf{Common Name} & \textbf{Obs.} & \textbf{Notes} \\
\midrule
\endhead
\bottomrule
\endfoot
\multicolumn{4}{l}{\textit{Food Plants}} \\
\textit{Pinus monophylla} & Singleleaf pinyon & 36+ & Abundant, woodland belt \\
\textit{Neltuma glandulosa} & Honey mesquite & 9 & Valley floors \\
\textit{Salvia columbariae} & Chia & 48 & Research grade \\
\textit{Eriocoma hymenoides} & Indian ricegrass & 12 & Sandy soils \\
\textit{Atriplex confertifolia} & Shadscale & + & Among 372 \textit{Atriplex} obs. \\
\textit{Atriplex hymenelytra} & Desert holly & + & Research grade \\
\textit{Lomatium mohavense} & Mojave desertparsley & 28 & Local species \\
\textit{Typha domingensis} & Southern cattail & 14 & Springs, seeps \\
\textit{Yucca brevifolia} & Joshua tree & 37 & Endemic, research grade \\
\textit{Quercus} spp. & Oaks & 0 & Trade item from western Sierra \\
\midrule
\multicolumn{4}{l}{\textit{Medicinal Plants}} \\
\textit{Artemisia tridentata} & Big sagebrush & 74 & Multiple subspecies \\
\textit{Larrea tridentata} & Creosote bush & 74 & Valley floors \\
\textit{Salix lasiolepis} & Arroyo willow & 79 & Primary willow species \\
\textit{Ephedra nevadensis} & Mormon tea & 3+ & Research grade \\
\textit{Juniperus osteosperma} & Utah juniper & 12 & Inyo Mountains \\
\textit{Eriodictyon} spp. & Yerba santa & 0 & Trade item from western Sierra \\
\midrule
\multicolumn{4}{l}{\textit{Ceremonial Plants}} \\
\textit{Datura wrightii} & Sacred datura & 33 & Research grade \\
\textit{Nicotiana obtusifolia} & Desert tobacco & 66 & Native, cultivated \\
\textit{Salvia apiana} & White sage & 0 & Trade item from south \\
\bottomrule
\end{longtable}
\end{small}

\noindent Data source: iNaturalist (\url{https://www.inaturalist.org}), research-grade observations prioritized.

\end{document}
